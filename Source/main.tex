\documentclass[openany, a4paper, 10pt]{book}
%\usepackage{ctex}
\usepackage{bm}
%\usepackage[fleqn]{amsmath}
\usepackage{harpoon}
\usepackage{fontspec}
\usepackage{listings}
\usepackage[left=1.5cm, right=1.5cm]{geometry}
\usepackage{setspace}
\usepackage{bm}
\usepackage{cmap}
\usepackage{ctex}
\usepackage{cite}
\usepackage{float}
\usepackage{xeCJK}
\usepackage{amsthm}
\usepackage{amsmath}
\usepackage{amssymb}
\usepackage{setspace}
\usepackage{enumerate}
\usepackage{indentfirst}
\allowdisplaybreaks
\oddsidemargin -0.1 true cm
\if@twoside
	\evensidemargin -0.1 true cm
\fi
%\setlength{\parindent}{0em}
%\setlength{\mathindent}{0pt}
\newfontfamily\Courier{Courier New}
\lstset{
	language=C++,
	tabsize=4,
	breaklines=tr,
	extendedchars=false
	xleftmargin=0em,
	xrightmargin=0em,
	aboveskip=1em,
	numberstyle=\small\Courier,
    basicstyle=\small\Courier
}
\begin{document}
	\title{\textbf{\LARGE{Standard Code Library}}}
	\author{Shanghai Jiao Tong University}
	\date{October, 2015}
	\maketitle
	\tableofcontents
	\begin{spacing}{1.1}
	\chapter{数论算法}
		\section{快速数论变换}
			使用条件及注意事项:$mod$必须要是一个形如$a2^b + 1$的数,$prt$表示$mod$的原根。
			\begin{lstlisting}
const int mod = 998244353;
const int prt = 3;
int prepare(int n) {
	int len = 1;
	for (; len <= 2 * n; len <<= 1);
	for (int i = 0; i <= len; i++) {
		e[0][i] = fpm(prt, (mod - 1) / len * i, mod);
		e[1][i] = fpm(prt, (mod - 1) / len * (len - i), mod);
	}
	return len;
}
void DFT(int *a, int n, int f) {
	for (int i = 0, j = 0; i < n; i++) {
		if (i > j) std::swap(a[i], a[j]);
		for (int t = n >> 1; (j ^= t) < t; t >>= 1);
	}
	for (int i = 2; i <= n; i <<= 1)
		for (int j = 0; j < n; j += i)
			for (int k = 0; k < (i >> 1); k++) {
				int A = a[j + k];
				int B = (long long)a[j + k + (i >> 1)] * e[f][n / i * k] % mod;
				a[j + k] = (A + B) % mod;
				a[j + k + (i >> 1)] = (A - B + mod) % mod;
			}
	if (f == 1) {
		long long rev = fpm(n, mod - 2, mod);
		for (int i = 0; i < n; i++) {
			a[i] = (long long)a[i] * rev % mod;
		}
	}
}
\end{lstlisting}

		\section{多项式求逆}
			使用条件及注意事项:求一个多项式在模意义下的逆元。
			\begin{lstlisting}
void getInv(int *a, int *b, int n) {
	static int tmp[MAXN];
	std::fill(b, b + n, 0);
	b[0] = fpm(a[0], mod - 2, mod);
	for (int c = 1; c <= n; c <<= 1) {
		for (int i = 0; i < c; i++) tmp[i] = a[i];
		std::fill(b + c, b + (c << 1), 0);
		std::fill(tmp + c, tmp + (c << 1), 0);
		DFT(tmp, c << 1, 0);
		DFT(b, c << 1, 0);
		for (int i = 0; i < (c << 1); i++) {
			b[i] = (long long)(2 - (long long)tmp[i] * b[i] % mod + mod) * b[i] % mod;
		}
		DFT(b, c << 1, 1);
		std::fill(b + c, b + (c << 1), 0);
	}
}
\end{lstlisting}

		\section{中国剩余定理}
			使用条件及注意事项:模数可以不互质。
			\begin{lstlisting}
bool solve(int n, std::pair<long long, long long> input[],
                  std::pair<long long, long long> &output) {
    output = std::make_pair(1, 1);
    for (int i = 0; i < n; ++i) {
        long long number, useless;
        euclid(output.second, input[i].second, number, useless);
        long long divisor = std::__gcd(output.second, input[i].second);
        if ((input[i].first - output.first) % divisor) {
            return false;
        }
        number *= (input[i].first - output.first) / divisor;
        fix(number, input[i].second);
        output.first += output.second * number;
        output.second *= input[i].second / divisor;
        fix(output.first, output.second);
    }
    return true;
}
\end{lstlisting}

		\section{Miller Rabin}
			\begin{lstlisting}
const int BASE[12] = {2, 3, 5, 7, 11, 13, 17, 19, 23, 29, 31, 37};

bool check(const long long &prime, const long long &base) {
    long long number = prime - 1;
    for (; ~number & 1; number >>= 1);
    long long result = power_mod(base, number, prime);
    for (; number != prime - 1 && result != 1 && result != prime - 1; number <<= 1) {
        result = multiply_mod(result, result, prime);
    }
    return result == prime - 1 || (number & 1) == 1;
}

bool miller_rabin(const long long &number) {
    if (number < 2) {
        return false;
    }
    if (number < 4) {
        return true;
    }
    if (~number & 1) {
        return false;
    }
    for (int i = 0; i < 12 && BASE[i] < number; ++i) {
        if (!check(number, BASE[i])) {
            return false;
        }
    }
    return true;
}
\end{lstlisting}

		\section{Pollard Rho}
			\begin{lstlisting}
long long pollard_rho(const long long &number, const long long &seed) {
    long long x = rand() % (number - 1) + 1, y = x;
    for (int head = 1, tail = 2; ; ) {
        x = multiply_mod(x, x, number);
        x = add_mod(x, seed, number);
        if (x == y) {
            return number;
        }
        long long answer = std::__gcd(abs(x - y), number);
        if (answer > 1 && answer < number) {
            return answer;
        }
        if (++head == tail) {
            y = x;
            tail <<= 1;
        }
    }
}
void factorize(const long long &number, std::vector<long long> &divisor) {
    if (number > 1) {
        if (miller_rabin(number)) {
            divisor.push_back(number);
        } else {
            long long factor = number;
            for (; factor >= number;
                   factor = pollard_rho(number, rand() % (number - 1) + 1));
            factorize(number / factor, divisor);
            factorize(factor, divisor);
        }
    }
}
\end{lstlisting}

		\section{坚固的逆元}
			\begin{lstlisting}
long long inverse(const long long &x, const long long &mod) {
    if (x == 1) {
        return 1;
    } else {
        return (mod - mod / x) * inverse(mod % x, mod) % mod;
    }
}
\end{lstlisting}

		\section{直线下整点个数}
			\begin{lstlisting}
long long solve(const long long &n, const long long &a,
                const long long &b, const long long &m) {
    if (b == 0) {
        return n * (a / m);
    }
    if (a >= m) {
        return n * (a / m) + solve(n, a % m, b, m);
    }
    if (b >= m) {
        return (n - 1) * n / 2 * (b / m) + solve(n, a, b % m, m);
    }
    return solve((a + b * n) / m, (a + b * n) % m, m, b);
}
\end{lstlisting}

	\chapter{数值算法}
		\section{快速傅立叶变换}
			\begin{lstlisting}
int prepare(int n) {
	int len = 1;
	for (; len <= 2 * n; len <<= 1);
	for (int i = 0; i < len; i++) {
		e[0][i] = Complex(cos(2 * pi * i / len), sin(2 * pi * i / len));
		e[1][i] = Complex(cos(2 * pi * i / len), -sin(2 * pi * i / len));
	}
	return len;
}
void DFT(Complex *a, int n, int f) {
	for (int i = 0, j = 0; i < n; i++) {
		if (i > j) std::swap(a[i], a[j]);
		for (int t = n >> 1; (j ^= t) < t; t >>= 1);
	}
	for (int i = 2; i <= n; i <<= 1)
		for (int j = 0; j < n; j += i)
			for (int k = 0; k < (i >> 1); k++) {
				Complex A = a[j + k];
				Complex B = e[f][n / i * k] * a[j + k + (i >> 1)];
				a[j + k] = A + B;
				a[j + k + (i >> 1)] = A - B;
			}
	if (f == 1) {
		for (int i = 0; i < n; i++)
			a[i].a /= n;
	}
}
\end{lstlisting}

		\section{单纯形法求解线性规划}
			使用条件及注意事项:返回结果为$max\{c_{1 \times m} \cdot x_{m \times 1} \ | \ x_{m \times 1} \geq 0_{m \times 1}, a_{n \times m} \cdot x_{m \times 1} \leq b_{n \times 1}\}$
			\begin{lstlisting}
std::vector<double> solve(const std::vector<std::vector<double> > &a, 
                          const std::vector<double> &b, const std::vector<double> &c) {
    int n = (int)a.size(), m = (int)a[0].size() + 1;
    std::vector<std::vector<double> > value(n + 2, std::vector<double>(m + 1));
    std::vector<int> index(n + m);
    int r = n, s = m - 1;
    for (int i = 0; i < n + m; ++i) index[i] = i;
    for (int i = 0; i < n; ++i) {
        for (int j = 0; j < m - 1; ++j)
            value[i][j] = -a[i][j];
        value[i][m - 1] = 1;
        value[i][m] = b[i];
        if (value[r][m] > value[i][m]) r = i;
    }
    for (int j = 0; j < m - 1; ++j) value[n][j] = c[j];
    value[n + 1][m - 1] = -1;
    for (double number; ; ) {
        if (r < n) {
            std::swap(index[s], index[r + m]);
            value[r][s] = 1 / value[r][s];
            for (int j = 0; j <= m; ++j)
                if (j != s) value[r][j] *= -value[r][s];
            for (int i = 0; i <= n + 1; ++i) {
                if (i != r) {
                    for (int j = 0; j <= m; ++j)
                        if (j != s) value[i][j] += value[r][j] * value[i][s];
                    value[i][s] *= value[r][s];
                }
            }
        }
        r = s = -1;
        for (int j = 0; j < m; ++j) {
            if (s < 0 || index[s] > index[j]) {
                if (value[n + 1][j] > eps || value[n + 1][j] > -eps && value[n][j] > eps) {
                    s = j;
                }
            }
        }
        if (s < 0) break;
        for (int i = 0; i < n; ++i) {
            if (value[i][s] < -eps) {
                if (r < 0
                || (number = value[r][m] / value[r][s] - value[i][m] / value[i][s]) < -eps
                || number < eps && index[r + m] > index[i + m]) {
                     r = i;
                }
            }
        }
        if (r < 0) {
            //    Solution is unbounded.
            return std::vector<double>();
        }
    }
    if (value[n + 1][m] < -eps) {
        //    No solution.
        return std::vector<double>();
    }
    std::vector<double> answer(m - 1);
    for (int i = m; i < n + m; ++i)
        if (index[i] < m - 1) answer[index[i]] = value[i - m][m];
    return answer;
}
\end{lstlisting}

		\section{自适应辛普森}
			\begin{lstlisting}
double area(const double &left, const double &right) {
    double mid = (left + right) / 2;
    return (right - left) * (calc(left) + 4 * calc(mid) + calc(right)) / 6;
}

double simpson(const double &left, const double &right,
               const double &eps, const double &area_sum) {
    double mid = (left + right) / 2;
    double area_left = area(left, mid);
    double area_right = area(mid, right);
    double area_total = area_left + area_right;
    if (std::abs(area_total - area_sum) < 15 * eps) {
        return area_total + (area_total - area_sum) / 15;
    }
    return simpson(left, mid, eps / 2, area_left)
         + simpson(mid, right, eps / 2, area_right);
}

double simpson(const double &left, const double &right, const double &eps) {
    return simpson(left, right, eps, area(left, right));
}
\end{lstlisting}

	\chapter{数据结构}
		\section{Splay普通操作版}
		\section{Splay区间操作版}
		\section{坚固的Treap}
		\section{k-d树}
		\section{树链剖分}
		\section{Link-Cut-Tree}
	\chapter{图论}
		\section{强连通分量}
			\begin{lstlisting}
int stamp, comps, top;
int dfn[N], low[N], comp[N], stack[N];

void tarjan(int x) {
    dfn[x] = low[x] = ++stamp;
    stack[top++] = x;
    for (int i = 0; i < (int)edge[x].size(); ++i) {
        int y = edge[x][i];
        if (!dfn[y]) {
            tarjan(y);
            low[x] = std::min(low[x], low[y]);
        } else if (!comp[y]) {
            low[x] = std::min(low[x], dfn[y]);
        }
    }
    if (low[x] == dfn[x]) {
        comps++;
        do {
            int y = stack[--top];
            comp[y] = comps;
        } while (stack[top] != x);
    }
}

void solve() {
    stamp = comps = top = 0;
    std::fill(dfn, dfn + n, 0);
    std::fill(comp, comp + n, 0);
    for (int i = 0; i < n; ++i) {
        if (!dfn[i]) {
            tarjan(i);
        }
    }
}
\end{lstlisting}

		\section{2-SAT问题}
			\begin{lstlisting}
int stamp, comps, top;
int dfn[N], low[N], comp[N], stack[N];
void add(int x, int a, int y, int b) {
    edge[x << 1 | a].push_back(y << 1 | b);
}
void tarjan(int x) {
    dfn[x] = low[x] = ++stamp;
    stack[top++] = x;
    for (int i = 0; i < (int)edge[x].size(); ++i) {
        int y = edge[x][i];
        if (!dfn[y]) {
            tarjan(y);
            low[x] = std::min(low[x], low[y]);
        } else if (!comp[y]) {
            low[x] = std::min(low[x], dfn[y]);
        }
    }
    if (low[x] == dfn[x]) {
        comps++;
        do {
            int y = stack[--top];
            comp[y] = comps;
        } while (stack[top] != x);
    }
}
bool solve() {
    int counter = n + n + 1;
    stamp = top = comps = 0;
    std::fill(dfn, dfn + counter, 0);
    std::fill(comp, comp + counter, 0);
    for (int i = 0; i < counter; ++i) {
        if (!dfn[i]) {
            tarjan(i);
        }
    }
    for (int i = 0; i < n; ++i) {
        if (comp[i << 1] == comp[i << 1 | 1]) {
            return false;
        }
        answer[i] = (comp[i << 1 | 1] < comp[i << 1]);
    }
    return true;
}
\end{lstlisting}

		\section{二分图最大匹配}
			\subsection{Hungary算法}
				时间复杂度:$\mathcal{O}(V \cdot E)$
				\begin{lstlisting}
int n, m, stamp;
int match[N], visit[N];

bool dfs(int x) {
    for (int i = 0; i < (int)edge[x].size(); ++i) {
        int y = edge[x][i];
        if (visit[y] != stamp) {
            visit[y] = stamp;
            if (match[y] == -1 || dfs(match[y])) {
                match[y] = x;
                return true;
            }
        }
    }
    return false;
}

int solve() {
    std::fill(match, match + m, -1);
    int answer = 0;
    for (int i = 0; i < n; ++i) {
        stamp++;
        answer += dfs(i);
    }
    return answer;
}
\end{lstlisting}

			\subsection{Hopcroft Karp算法}
				时间复杂度:$\mathcal{O}(\sqrt{V} \cdot E)$
				\begin{lstlisting}
int matchx[N], matchy[N], level[N];
bool dfs(int x) {
    for (int i = 0; i < (int)edge[x].size(); ++i) {
        int y = edge[x][i];
        int w = matchy[y];
        if (w == -1 || level[x] + 1 == level[w] && dfs(w)) {
            matchx[x] = y;
            matchy[y] = x;
            return true;
        }
    }
    level[x] = -1;
    return false;
}
int solve() {
    std::fill(matchx, matchx + n, -1);
    std::fill(matchy, matchy + m, -1);
    for (int answer = 0; ; ) {
        std::vector<int> queue;
        for (int i = 0; i < n; ++i) {
            if (matchx[i] == -1) {
                level[i] = 0;
                queue.push_back(i);
            } else {
                level[i] = -1;
            }
        }
        for (int head = 0; head < (int)queue.size(); ++head) {
            int x = queue[head];
            for (int i = 0; i < (int)edge[x].size(); ++i) {
                int y = edge[x][i];
                int w = matchy[y];
                if (w != -1 && level[w] < 0) {
                    level[w] = level[x] + 1;
                    queue.push_back(w);
                }
            }
        }
        int delta = 0;
        for (int i = 0; i < n; ++i) {
            if (matchx[i] == -1 && dfs(i)) {
                delta++;
            }
        }
        if (delta == 0) {
            return answer;
        } else {
            answer += delta;
        }
    }
}
\end{lstlisting}

		\section{二分图最大权匹配}
			时间复杂度:$\mathcal{O}(V^4)$
			\begin{lstlisting}
int labelx[N], labely[N], match[N], slack[N];
bool visitx[N], visity[N];

bool dfs(int x) {
    visitx[x] = true;
    for (int y = 0; y < n; ++y) {
        if (visity[y]) {
            continue;
        }
        int delta = labelx[x] + labely[y] - graph[x][y];
        if (delta == 0) {
            visity[y] = true;
            if (match[y] == -1 || dfs(match[y])) {
                match[y] = x;
                return true;
            }
        } else {
            slack[y] = std::min(slack[y], delta);
        }
    }
    return false;
}

int solve() {
    for (int i = 0; i < n; ++i) {
        match[i] = -1;
        labelx[i] = INT_MIN;
        labely[i] = 0;
        for (int j = 0; j < n; ++j) {
            labelx[i] = std::max(labelx[i], graph[i][j]);
        }
    }
    for (int i = 0; i < n; ++i) {
        while (true) {
            std::fill(visitx, visitx + n, 0);
            std::fill(visity, visity + n, 0);
            for (int j = 0; j < n; ++j) {
                slack[j] = INT_MAX;
            }
            if (dfs(i)) {
                break;
            }
            int delta = INT_MAX;
            for (int j = 0; j < n; ++j) {
                if (!visity[j]) {
                    delta = std::min(delta, slack[j]);
                }
            }
            for (int j = 0; j < n; ++j) {
                if (visitx[j]) {
                    labelx[j] -= delta;
                }
                if (visity[j]) {
                    labely[j] += delta;
                } else {
                    slack[j] -= delta;
                }
            }
        }
    }
    int answer = 0;
    for (int i = 0; i < n; ++i) {
        answer += graph[match[i]][i];
    }
    return answer;
}
\end{lstlisting}

		\section{最大流}
			时间复杂度:$\mathcal{O}(V^2 \cdot E)$
			\begin{lstlisting}
struct EdgeList {
    int size;
    int last[N];
    int succ[M], other[M], flow[M];
    void clear(int n) {
        size = 0;
        fill(last, last + n, -1);
    }
    void add(int x, int y, int c) {
        succ[size] = last[x];
        last[x] = size;
        other[size] = y;
        flow[size++] = c;
    }
} e;

int n, source, target;
int dist[N], curr[N];

void add(int x, int y, int c) {
    e.add(x, y, c);
    e.add(y, x, 0);
}

bool relabel() {
    std::vector<int> queue;
    for (int i = 0; i < n; ++i) {
        curr[i] = e.last[i];
        dist[i] = -1;
    }
    queue.push_back(target);
    dist[target] = 0;
    for (int head = 0; head < (int)queue.size(); ++head) {
        int x = queue[head];
        for (int i = e.last[x]; ~i; i = e.succ[i]) {
            int y = e.other[i];
            if (e.flow[i ^ 1] && dist[y] == -1) {
                dist[y] = dist[x] + 1;
                queue.push_back(y);
            }
        }
    }
    return ~dist[source];
}

int dfs(int x, int answer) {
    if (x == target) {
        return answer;
    }
    int delta = answer;
    for (int &i = curr[x]; ~i; i = e.succ[i]) {
        int y = e.other[i];
        if (e.flow[i] && dist[x] == dist[y] + 1) {
            int number = dfs(y, std::min(e.flow[i], delta));
            e.flow[i] -= number;
            e.flow[i ^ 1] += number;
            delta -= number;
        }
        if (delta == 0) {
            break;
        }
    }
    return answer - delta;
}

int solve() {
    int answer = 0;
    while (relabel()) {
        answer += dfs(source, INT_MAX));
    }
    return answer;
}
\end{lstlisting}

		\section{上下界网络流}
			$B(u,v)$表示边$(u,v)$流量的下界,$C(u,v)$表示边$(u,v)$流量的上界,$F(u,v)$表示边$(u,v)$的流量。
			设$G(u,v) = F(u,v) - B(u,v)$,显然有
			$$0 \leq G(u,v) \leq C(u,v)-B(u,v)$$
		\subsection{无源汇的上下界可行流}
			建立超级源点$S^*$和超级汇点$T^*$,对于原图每条边$(u,v)$在新网络中连如下三条边:$S^* \rightarrow v$,容量为$B(u,v)$;$u \rightarrow T^*$,容量为$B(u,v)$;$u \rightarrow v$,容量为$C(u,v) - B(u,v)$。最后求新网络的最大流,判断从超级源点$S^*$出发的边是否都满流即可,边$(u,v)$的最终解中的实际流量为$G(u,v)+B(u,v)$。
		\subsection{有源汇的上下界可行流}
			从汇点$T$到源点$S$连一条上界为$\infty$,下界为$0$的边。按照\textbf{无源汇的上下界可行流}一样做即可,流量即为$T \rightarrow S$边上的流量。
		\subsection{有源汇的上下界最大流}
			\begin{enumerate}
				\item 在\textbf{有源汇的上下界可行流}中,从汇点$T$到源点$S$的边改为连一条上界为$\infty$,下届为$x$的边。$x$满足二分性质,找到最大的$x$使得新网络存在\textbf{无源汇的上下界可行流}即为原图的最大流。
				\item 从汇点$T$到源点$S$连一条上界为$\infty$,下界为$0$的边,变成无源汇的网络。按照\textbf{无源汇的上下界可行流}的方法,建立超级源点$S^*$和超级汇点$T^*$,求一遍$S^* \rightarrow T^*$的最大流,再将从汇点$T$到源点$S$的这条边拆掉,求一次$S \rightarrow T$的最大流即可。
			\end{enumerate}
		\subsection{有源汇的上下界最小流}
			\begin{enumerate}
				\item 在\textbf{有源汇的上下界可行流}中,从汇点$T$到源点$S$的边改为连一条上界为$x$,下界为$0$的边。$x$满足二分性质,找到最小的$x$使得新网络存在\textbf{无源汇的上下界可行流}即为原图的最小流。
				\item 按照\textbf{无源汇的上下界可行流}的方法,建立超级源点$S^*$与超级汇点$T^*$,求一遍$S^* \rightarrow T^*$的最大流,但是注意这一次不加上汇点$T$到源点$S$的这条边,即不使之改为无源汇的网络去求解。求完后,再加上那条汇点$T$到源点$S$上界$\infty$的边。因为这条边下界为$0$,所以$S^*$,$T^*$无影响,再直接求一次$S^* \rightarrow T^*$的最大流。若超级源点$S^*$出发的边全部满流,则$T \rightarrow S$边上的流量即为原图的最小流,否则无解。
			\end{enumerate}
		\section{最小费用最大流}
		\subsection{稀疏图}
			时间复杂度:$\mathcal{O}(V \cdot E^2)$
			\begin{lstlisting}
struct EdgeList {
    int size;
    int last[N];
    int succ[M], other[M], flow[M], cost[M];
    void clear(int n) {
        size = 0;
        std::fill(last, last + n, -1);
    }
    void add(int x, int y, int c, int w) {
        succ[size] = last[x];
        last[x] = size;
        other[size] = y;
        flow[size] = c;
        cost[size++] = w;
    }
} e;

int n, source, target;
int prev[N];

void add(int x, int y, int c, int w) {
    e.add(x, y, c, w);
    e.add(y, x, 0, -w);
}

bool augment() {
    static int dist[N], occur[N];
    std::vector<int> queue;
    std::fill(dist, dist + n, INT_MAX);
    std::fill(occur, occur + n, 0);
    dist[source] = 0;
    occur[source] = true;
    queue.push_back(source);
    for (int head = 0; head < (int)queue.size(); ++head) {
        int x = queue[head];
        for (int i = e.last[x]; ~i; i = e.succ[i]) {
            int y = e.other[i];
            if (e.flow[i] && dist[y] > dist[x] + e.cost[i]) {
                dist[y] = dist[x] + e.cost[i];
                prev[y] = i;
                if (!occur[y]) {
                    occur[y] = true;
                    queue.push_back(y);
                }
            }
        }
        occur[x] = false;
    }
    return dist[target] < INT_MAX;
}

std::pair<int, int> solve() {
    std::pair<int, int> answer = std::make_pair(0, 0);
    while (augment()) {
        int number = INT_MAX;
        for (int i = target; i != source; i = e.other[prev[i] ^ 1]) {
            number = std::min(number, e.flow[prev[i]]);
        }
        answer.first += number;
        for (int i = target; i != source; i = e.other[prev[i] ^ 1]) {
            e.flow[prev[i]] -= number;
            e.flow[prev[i] ^ 1] += number;
            answer.second += number * e.cost[prev[i]];
        }
    }
    return answer;
}
\end{lstlisting}

		\subsection{稠密图}
			使用条件:费用非负\\
			\indent 时间复杂度:$\mathcal{O}(V \cdot E^2)$
			\begin{lstlisting}
struct EdgeList {
    int size;
    int last[N];
    int succ[M], other[M], flow[M], cost[M];
    void clear(int n) {
        size = 0;
        std::fill(last, last + n, -1);
    }
    void add(int x, int y, int c, int w) {
        succ[size] = last[x];
        last[x] = size;
        other[size] = y;
        flow[size] = c;
        cost[size++] = w;
    }
} e;

int n, source, target, flow, cost;
int slack[N], dist[N];
bool visit[N];

void add(int x, int y, int c, int w) {
    e.add(x, y, c, w);
    e.add(y, x, 0, -w);
}

bool relabel() {
    int delta = INT_MAX;
    for (int i = 0; i < n; ++i) {
        if (!visit[i]) {
            delta = std::min(delta, slack[i]);
        }
        slack[i] = INT_MAX;
    }
    if (delta == INT_MAX) {
        return true;
    }
    for (int i = 0; i < n; ++i) {
        if (visit[i]) {
            dist[i] += delta;
        }
    }
    return false;
}

int dfs(int x, int answer) {
    if (x == target) {
        flow += answer;
        cost += answer * (dist[source] - dist[target]);
        return answer;
    }
    visit[x] = true;
    int delta = answer;
    for (int i = e.last[x]; ~i; i = e.succ[i]) {
        int y = e.other[i];
        if (e.flow[i] > 0 && !visit[y]) {
            if (dist[y] + e.cost[i] == dist[x]) {
                int number = dfs(y, std::min(e.flow[i], delta));
                e.flow[i] -= number;
                e.flow[i ^ 1] += number;
                delta -= number;
                if (delta == 0) {
                    dist[x] = INT_MIN;
                    return answer;
                }
            } else {
                slack[y] = std::min(slack[y], dist[y] + e.cost[i] - dist[x]);
            }
        }
    }
    return answer - delta;
}

std::pair<int, int> solve() {
    flow = cost = 0;
    std::fill(dist, dist + n, 0);
    do {
        do {
            fill(visit, visit + n, 0);
        } while (dfs(source, INT_MAX));
    } while (!relabel());
    return std::make_pair(flow, cost);
}
\end{lstlisting}

		\section{一般图最大匹配}
			时间复杂度:$\mathcal{O}(V^3)$
			\begin{lstlisting}
int match[N], belong[N], next[N], mark[N], visit[N];
std::vector<int> queue;

int find(int x) {
    if (belong[x] != x) {
        belong[x] = find(belong[x]);
    }
    return belong[x];
}

void merge(int x, int y) {
    x = find(x);
    y = find(y);
    if (x != y) {
        belong[x] = y;
    }
}

int lca(int x, int y) {
    static int stamp = 0;
    stamp++;
    while (true) {
        if (x != -1) {
            x = find(x);
            if (visit[x] == stamp) {
                return x;
            }
            visit[x] = stamp;
            if (match[x] != -1) {
                x = next[match[x]];
            } else {
                x = -1;
            }
        }
        std::swap(x, y);
    }
}

void group(int a, int p) {
    while (a != p) {
        int b = match[a], c = next[b];
        if (find(c) != p) {
            next[c] = b;
        }
        if (mark[b] == 2) {
            mark[b] = 1;
            queue.push_back(b);
        }
        if (mark[c] == 2) {
            mark[c] = 1;
            queue.push_back(c);
        }
        merge(a, b);
        merge(b, c);
        a = c;
    }
}

void augment(int source) {
    queue.clear();
    for (int i = 0; i < n; ++i) {
        next[i] = visit[i] = -1;
        belong[i] = i;
        mark[i] = 0;
    }
    mark[source] = 1;
    queue.push_back(source);
    for (int head = 0; head < (int)queue.size() && match[source] == -1; ++head) {
        int x = queue[head];
        for (int i = 0; i < (int)edge[x].size(); ++i) {
            int y = edge[x][i];
            if (match[x] == y || find(x) == find(y) || mark[y] == 2) {
                continue;
            }
            if (mark[y] == 1) {
                int r = lca(x, y);
                if (find(x) != r) {
                    next[x] = y;
                }
                if (find(y) != r) {
                    next[y] = x;
                }
                group(x, r);
                group(y, r);
            } else if (match[y] == -1) {
                next[y] = x;
                for (int u = y; u != -1; ) {
                    int v = next[u];
                    int mv = match[v];
                    match[v] = u;
                    match[u] = v;
                    u = mv;
                }
                break;
            } else {
                next[y] = x;
                mark[y] = 2;
                mark[match[y]] = 1;
                queue.push_back(match[y]);
            }
        }
    }
}

int solve() {
    std::fill(match, match + n, -1);
    for (int i = 0; i < n; ++i) {
        if (match[i] == -1) {
            augment(i);
        }
    }
    int answer = 0;
    for (int i = 0; i < n; ++i) {
        answer += (match[i] != -1);
    }
    return answer;
}
\end{lstlisting}

		\section{无向图全局最小割}
			时间复杂度:$\mathcal{O}(V^3)$\\
			\indent 注意事项:处理重边时,应该对边权累加
			\begin{lstlisting}
int node[N], dist[N];
bool visit[N];
int solve(int n) {
    int answer = INT_MAX;
    for (int i = 0; i < n; ++i) {
        node[i] = i;
    }
    while (n > 1) {
        int max = 1;
        for (int i = 0; i < n; ++i) {
            dist[node[i]] = graph[node[0]][node[i]];
            if (dist[node[i]] > dist[node[max]]) {
                max = i;
            }
        }
        int prev = 0;
        memset(visit, 0, sizeof(visit));
        visit[node[0]] = true;
        for (int i = 1; i < n; ++i) {
            if (i == n - 1) {
                answer = std::min(answer, dist[node[max]]);
                for (int k = 0; k < n; ++k) {
                    graph[node[k]][node[prev]] =
                        (graph[node[prev]][node[k]] += graph[node[k]][node[max]]);
                }
                node[max] = node[--n];
            }
            visit[node[max]] = true;
            prev = max;
            max = -1;
            for (int j = 1; j < n; ++j) {
                if (!visit[node[j]]) {
                    dist[node[j]] += graph[node[prev]][node[j]];
                    if (max == -1 || dist[node[max]] < dist[node[j]]) {
                        max = j;
                    }
                }
            }
        }
    }
    return answer;
}
\end{lstlisting}

		%\subsection{最小树形图}
		\section{有根树的同构}
			时间复杂度:$\mathcal{O}(V log V)$
			\begin{lstlisting}
const unsigned long long MAGIC = 4423;
unsigned long long magic[N];
std::pair<unsigned long long, int> hash[N];
void solve(int root) {
    magic[0] = 1;
    for (int i = 1; i <= n; ++i) {
        magic[i] = magic[i - 1] * MAGIC;
    }
    std::vector<int> queue;
    queue.push_back(root);
    for (int head = 0; head < (int)queue.size(); ++head) {
        int x = queue[head];
        for (int i = 0; i < (int)son[x].size(); ++i) {
            int y = son[x][i];
            queue.push_back(y);
        }
    }
    for (int index = n - 1; index >= 0; --index) {
        int x = queue[index];
        hash[x] = std::make_pair(0, 0);

        std::vector<std::pair<unsigned long long, int> > value;
        for (int i = 0; i < (int)son[x].size(); ++i) {
            int y = son[x][i];
            value.push_back(hash[y]);
        }
        std::sort(value.begin(), value.end());
        
        hash[x].first = hash[x].first * magic[1] + 37;
        hash[x].second++;
        for (int i = 0; i < (int)value.size(); ++i) {
            hash[x].first = hash[x].first * magic[value[i].second] + value[i].first;
            hash[x].second += value[i].second;
        }
        hash[x].first = hash[x].first * magic[1] + 41;
        hash[x].second++;
    }
}
\end{lstlisting}

		%\subsection{度限制生成树}
		%\subsection{弦图相关}
		%\subsubsection{弦图的判定}
		%\subsubsection{弦图的团数}
		\section{哈密尔顿回路(ORE性质的图)}
			ORE性质:$$\forall x,y \in V \wedge (x,y) \notin E \ \ s.t. \ \ deg_x+deg_y \geq n$$
			\indent 返回结果:从顶点$1$出发的一个哈密尔顿回路\\
			\indent 使用条件:$n \geq 3$
			\begin{lstlisting}
int left[N], right[N], next[N], last[N];

void cover(int x) {
    left[right[x]] = left[x];
    right[left[x]] = right[x];
}

int adjacent(int x) {
    for (int i = right[0]; i <= n; i = right[i]) {
        if (graph[x][i]) {
            return i;
        }
    }
    return 0;
}

std::vector<int> solve() {
    for (int i = 1; i <= n; ++i) {
        left[i] = i - 1;
        right[i] = i + 1;
    }
    int head, tail;
    for (int i = 2; i <= n; ++i) {
        if (graph[1][i]) {
            head = 1;
            tail = i;
            cover(head);
            cover(tail);
            next[head] = tail;
            break;
        }
    }
    while (true) {
        int x;
        while (x = adjacent(head)) {
            next[x] = head;
            head = x;
            cover(head);
        }
        while (x = adjacent(tail)) {
            next[tail] = x;
            tail = x;
            cover(tail);
        }
        if (!graph[head][tail]) {
            for (int i = head, j; i != tail; i = next[i]) {
                if (graph[head][next[i]] && graph[tail][i]) {
                    for (j = head; j != i; j = next[j]) {
                        last[next[j]] = j;
                    }
                    j = next[head];
                    next[head] = next[i];
                    next[tail] = i;
                    tail = j;
                    for (j = i; j != head; j = last[j]) {
                        next[j] = last[j];
                    }
                    break;
                }
            }
        }
        next[tail] = head;
        if (right[0] > n) {
            break;
        }
        for (int i = head; i != tail; i = next[i]) {
            if (adjacent(i)) {
                head = next[i];
                tail = i;
                next[tail] = 0;
                break;
            }
        }
    }
    std::vector<int> answer;
    for (int i = head; ; i = next[i]) {
        if (i == 1) {
            answer.push_back(i);
            for (int j = next[i]; j != i; j = next[j]) {
                answer.push_back(j);
            }
            answer.push_back(i);
            break;
        }
        if (i == tail) {
            break;
        }
    }
    return answer;
}
\end{lstlisting}

	\chapter{字符串}
		\section{模式串匹配}
			\begin{lstlisting}
void build(char *pattern) {
    int length = (int)strlen(pattern + 1);
    fail[0] = -1;
    for (int i = 1, j; i <= length; ++i) {
        for (j = fail[i - 1]; j != -1 && pattern[i] != pattern[j + 1]; j = fail[j]);
        fail[i] = j + 1;
    }
}

void solve(char *text, char *pattern) {
    int length = (int)strlen(text + 1);
    for (int i = 1, j; i <= length; ++i) {
        for (j = match[i - 1]; j != -1 && text[i] != pattern[j + 1]; j = fail[j]);
        match[i] = j + 1;
    }
}
\end{lstlisting}

		\section{AC自动机}
			\begin{lstlisting}
int size, c[MAXT][26], f[MAXT], fail[MAXT], d[MAXT];
int alloc() {
	size++;
	std::fill(c[size], c[size] + 26, 0);
	f[size] = fail[size] = d[size] = 0;
	return size;
}
void insert(char *s) {
	int len = strlen(s + 1), p = 1;
	for (int i = 1; i <= len; i++) {
		if (c[p][s[i] - 'a']) p = c[p][s[i] - 'a'];
		else{
			int newnode = alloc();
			c[p][s[i] - 'a'] = newnode;
			d[newnode] = s[i] - 'a';
			f[newnode] = p;
			p = newnode;
		}
	}
}
void buildfail() {
	static int q[MAXT];
	int left = 0, right = 0;
	fail[1] = 0;
	for (int i = 0; i < 26; i++) {
		c[0][i] = 1;
		if (c[1][i]) q[++right] = c[1][i];
	}
	while (left < right) {
		left++;
		int p = fail[f[q[left]]];
		while (!c[p][d[q[left]]]) p = fail[p];
		fail[q[left]] = c[p][d[q[left]]];
		for (int i = 0; i < 26; i++) {
			if (c[q[left]][i]) {
				q[++right] = c[q[left]][i];
			}
		}
	}
	for (int i = 1; i <= size; i++)
		for (int j = 0; j < 26; j++) {
			int p = i;
			while (!c[p][j]) p = fail[p];
			c[i][j] = c[p][j];
		}
}
\end{lstlisting}

		\section{后缀数组}
			\begin{lstlisting}
namespace suffix_array{
	int wa[MAXN], wb[MAXN], ws[MAXN], wv[MAXN];
	bool cmp(int *r, int a, int b, int l) {
		return r[a] == r[b] && r[a + l] == r[b + l];
	}
	void DA(int *r, int *sa, int n, int m) {
		int *x = wa, *y = wb, *t;
		for (int i = 0; i < m; i++) ws[i] = 0;
		for (int i = 0; i < n; i++) ws[x[i] = r[i]]++;
		for (int i = 1; i < m; i++) ws[i] += ws[i - 1];
		for (int i = n - 1; i >= 0; i--) sa[--ws[x[i]]] = i;
		for (int i, j = 1, p = 1; p < n; j <<= 1, m = p) {
			for (p = 0, i = n - j; i < n; i++) y[p++] = i;
			for (i = 0; i < n; i++) if (sa[i] >= j) y[p++] = sa[i] - j;
			for (i = 0; i < n; i++) wv[i] = x[y[i]];
			for (i = 0; i < m; i++) ws[i] = 0;
			for (i = 0; i < n; i++) ws[wv[i]]++;
			for (i = 1; i < m; i++) ws[i] += ws[i-1];
			for (i = n - 1; i >= 0; i--) sa[--ws[wv[i]]] = y[i];
			for (t = x, x = y, y = t, p = 1, x[sa[0]] = 0, i = 1; i < n; i++)
				x[sa[i]] = cmp(y, sa[i - 1], sa[i], j) ? p - 1 : p++;
		}
	}
	void getheight(int *r, int *sa, int *rk, int *h, int n) {
		for (int i = 1; i <= n; i++) rk[sa[i]] = i;
		for (int i = 0, j, k = 0; i < n; h[rk[i++]] = k)
			for (k ? k-- : 0, j = sa[rk[i] - 1]; r[i + k] == r[j + k]; k++);
	}
};
\end{lstlisting}

		\section{广义后缀自动机}
			\begin{lstlisting}
void add(int x, int &last) {
	int lastnode = last;
	if (c[lastnode][x]) {
		int nownode = c[lastnode][x];
		if (l[nownode] == l[lastnode] + 1) last = nownode;
		else{
			int auxnode = ++size; l[auxnode] = l[lastnode] + 1;
			for (int i = 0; i < 26; i++) c[auxnode][i] = c[nownode][i];
			f[auxnode] = f[nownode]; f[nownode] = auxnode;
			for (; lastnode && c[lastnode][x] == nownode; lastnode = f[lastnode]) {
				c[lastnode][x] = auxnode;
			}
			last = auxnode;
		}
	}
	else{
		int newnode = ++size; l[newnode] = l[lastnode] + 1;
		for (; lastnode && !c[lastnode][x]; lastnode = f[lastnode]) c[lastnode][x] = newnode;
		if (!lastnode) f[newnode] = 1;
		else{
			int nownode = c[lastnode][x];
			if (l[lastnode] + 1 == l[nownode]) f[newnode] = nownode;
			else{
				int auxnode = ++size; l[auxnode] = l[lastnode] + 1;
				for (int i = 0; i < 26; i++) c[auxnode][i] = c[nownode][i];
				f[auxnode] = f[nownode]; f[nownode] = f[newnode] = auxnode;
				for (; lastnode && c[lastnode][x] == nownode; lastnode = f[lastnode]) {
					c[lastnode][x] = auxnode;
				}
			}
		}
		last = newnode;
	}
}
\end{lstlisting}

		\section{Manacher算法}
			\begin{lstlisting}
void manacher(char *text, int length) {
    palindrome[0] = 1;
    for (int i = 1, j = 0; i < length; ++i) {
        if (j + palindrome[j] <= i) {
            palindrome[i] = 0;
        } else {
            palindrome[i] = std::min(palindrome[(j << 1) - i], j + palindrome[j] - i);
        }
        while (i - palindrome[i] >= 0 && i + palindrome[i] < length 
                && text[i - palindrome[i]] == text[i + palindrome[i]]) {
            palindrome[i]++;
        }
        if (i + palindrome[i] > j + palindrome[j]) {
            j = i;
        }
    }
}
\end{lstlisting}

		\section{回文树}
			\begin{lstlisting}
struct Palindromic_Tree{
	int nTree, nStr, last, c[MAXT][26], fail[MAXT], r[MAXN], l[MAXN], s[MAXN];
	int allocate(int len) {
		l[nTree] = len;
		r[nTree] = 0;
		fail[nTree] = 0;
		memset(c[nTree], 0, sizeof(c[nTree]));
		return nTree++;
	}
	void init() {
		nTree = nStr = 0;
		int newEven = allocate(0);
		int newOdd = allocate(-1);
		last = newEven;
		fail[newEven] = newOdd;
		fail[newOdd] = newEven;
		s[0] = -1;
	}
	void add(int x) {
		s[++nStr] = x;
		int nownode = last;
		while (s[nStr - l[nownode] - 1] != s[nStr]) nownode = fail[nownode];
		if (!c[nownode][x]) {
			int newnode = allocate(l[nownode] + 2), &newfail = fail[newnode];
			newfail = fail[nownode];
			while (s[nStr - l[newfail] - 1] != s[nStr]) newfail = fail[newfail];
			newfail = c[newfail][x];
			c[nownode][x] = newnode;
		}
		last = c[nownode][x];
		r[last]++;
	}
	void count() {
		for (int i = nTree - 1; i >= 0; i--) {
			r[fail[i]] += r[i];
		}
	}
}
\end{lstlisting}

		\section{循环串最小表示}
			%\begin{lstlisting}
int solve(char *text, int length) {
    int i = 0, j = 1, delta = 0;
    while (i < length && j < length && delta < length) {
        char tokeni = text[(i + delta) % length];
        char tokenj = text[(j + delta) % length];
        if (tokeni == tokenj) {
            delta++;
        } else {
            if (tokeni > tokenj) {
                i += delta + 1;
            } else {
                j += delta + 1;
            }
            if (i == j) {
                j++;
            }
            delta = 0;
        }
    }
    return std::min(i, j);
}
\end{lstlisting}

	\chapter{计算几何}
		\section{二维基础}
			\subsection{点类}
				\begin{lstlisting}
struct Point{
	double x, y;
	Point() {}
	Point(double x, double y):x(x), y(y) {}
	Point operator +(const Point &p)const {
		return Point(x + p.x, y + p.y);
	}
	Point operator -(const Point &p)const {
		return Point(x - p.x, y - p.y);
	}
	Point operator *(const double &p)const {
		return Point(x * p, y * p);
	}
	Point operator /(const double &p)const {
		return Point(x / p, y / p);
	}
	int read() {
		return scanf("%lf%lf", &x, &y);
	}
};
struct Line{
	Point a, b;
	Line() {}
	Line(Point a, Point b):a(a), b(b) {}
};
\end{lstlisting}

			\subsection{凸包}
				\begin{lstlisting}
bool Pair_Comp(const Point &a, const Point &b) {
	if (dcmp(a.x - b.x) < 0) return true;
	if (dcmp(a.x - b.x) > 0) return false;
	return dcmp(a.y - b.y) < 0;
}
int Convex_Hull(int n, Point *P, Point *C) {
	sort(P, P + n, Pair_Comp);
	int top = 0;
	for (int i = 0; i < n; i++) {
		while (top >= 2 && dcmp(det(C[top - 1] - C[top - 2], P[i] - C[top - 2])) <= 0) top--;
		C[top++] = P[i];
	}
	int lasttop = top;
	for (int i = n - 1; i >= 0; i--) {
		while (top > lasttop && dcmp(det(C[top - 1] - C[top - 2], P[i] - C[top - 2])) <= 0) top--;
		C[top++] = P[i];
	}
	return top;
}
\end{lstlisting}

			\subsection{半平面交}
				\begin{lstlisting}
bool isOnLeft(const Point &x, const Line &l) {
	double d = det(x - l.a, l.b - l.a);
	return dcmp(d) <= 0;
}
// 传入一个线段的集合L,传出A,并且返回A的大小
int getIntersectionOfHalfPlane(int n, Line *L, Line *A) {
	Line *q = new Line[n + 1];
	Point *p = new Point[n + 1];
	sort(L, L + n, Polar_Angle_Comp_Line);
	int l = 1, r = 0;
	for (int i = 0; i < n; i++) {
		while (l < r && !isOnLeft(p[r - 1], L[i])) r--;
		while (l < r && !isOnLeft(p[l], L[i])) l++;
		q[++r] = L[i];
		if (l < r && is_Colinear(q[r], q[r - 1])) {
			r--;
			if (isOnLeft(L[i].a, q[r])) q[r] = L[i];
		}
		if (l < r) p[r - 1] = getIntersection(q[r - 1], q[r]);
	}
	while (l < r && !isOnLeft(p[r - 1], q[l])) r--;
	if (r - l + 1 <= 2) return 0;
	int tot = 0;
	for (int i = l; i <= r; i++) A[tot++] = q[i];
	return tot;
}
\end{lstlisting}

			\subsection{最近点对}
				
		\section{三维基础}
			\subsection{点类}
				\begin{lstlisting}
int dcmp(const double &x) {
	return fabs(x) < eps ? 0 : (x > 0 ? 1 : -1);
}

struct TPoint{
	double x, y, z;
	TPoint() {}
	TPoint(double x, double y, double z) : x(x), y(y), z(z) {}
	TPoint operator +(const TPoint &p)const {
		return TPoint(x + p.x, y + p.y, z + p.z);
	}
	TPoint operator -(const TPoint &p)const {
		return TPoint(x - p.x, y - p.y, z - p.z);
	}
	TPoint operator *(const double &p)const {
		return TPoint(x * p, y * p, z * p);
	}
	TPoint operator /(const double &p)const {
		return TPoint(x / p, y / p, z / p);
	}
	bool operator <(const TPoint &p)const {
		int dX = dcmp(x - p.x), dY = dcmp(y - p.y), dZ = dcmp(z - p.z);
		return dX < 0 || (dX == 0 && (dY < 0 || (dY == 0 && dZ < 0)));
	}
	bool read() {
		return scanf("%lf%lf%lf", &x, &y, &z) == 3;
	}
};

double sqrdist(const TPoint &a) {
	double ret = 0;
	ret += a.x * a.x;
	ret += a.y * a.y;
	ret += a.z * a.z;
	return ret;
}
double sqrdist(const TPoint &a, const TPoint &b) {
	double ret = 0;
	ret += (a.x - b.x) * (a.x - b.x);
	ret += (a.y - b.y) * (a.y - b.y);
	ret += (a.z - b.z) * (a.z - b.z);
	return ret;
}
double dist(const TPoint &a) {
	return sqrt(sqrdist(a));
}
double dist(const TPoint &a, const TPoint &b) {
	return sqrt(sqrdist(a, b));
}
TPoint det(const TPoint &a, const TPoint &b) {
	TPoint ret;
	ret.x = a.y * b.z - b.y * a.z;
	ret.y = a.z * b.x - b.z * a.x;
	ret.z = a.x * b.y - b.x * a.y;
	return ret;
}
double dot(const TPoint &a, const TPoint &b) {
	double ret = 0;
	ret += a.x * b.x;
	ret += a.y * b.y;
	ret += a.z * b.z;
	return ret;
}
double detdot(const TPoint &a, const TPoint &b, const TPoint &c, const TPoint &d) {
	return dot(det(b - a, c - a), d - a);
}
\end{lstlisting}

			\subsection{凸包}
				\begin{lstlisting}
struct Triangle{
	TPoint a, b, c;
	Triangle() {}
	Triangle(TPoint a, TPoint b, TPoint c) : a(a), b(b), c(c) {}
	double getArea() {
		TPoint ret = det(b - a, c - a);
		return dist(ret) / 2.0;
	}
};
namespace Convex_Hull {
	struct Face{
		int a, b, c;
		bool isOnConvex;
		Face() {}
		Face(int a, int b, int c) : a(a), b(b), c(c) {}
	};

	int nFace, left, right, whe[MAXN][MAXN];
	Face queue[MAXF], tmp[MAXF];

	bool isVisible(const std::vector<TPoint> &p, const Face &f, const TPoint &a) {
		return dcmp(detdot(p[f.a], p[f.b], p[f.c], a)) > 0;
	}

	bool init(std::vector<TPoint> &p) {
		bool check = false;
		for (int i = 1; i < (int)p.size(); i++) {
			if (dcmp(sqrdist(p[0], p[i]))) {
				std::swap(p[1], p[i]);
				check = true;
				break;
			}
		}
		if (!check) return false;
		check = false;
		for (int i = 2; i < (int)p.size(); i++) {
			if (dcmp(sqrdist(det(p[i] - p[0], p[1] - p[0])))) {
				std::swap(p[2], p[i]);
				check = true;
				break;
			}
		}
		if (!check) return false;
		check = false;
		for (int i = 3; i < (int)p.size(); i++) {
			if (dcmp(detdot(p[0], p[1], p[2], p[i]))) {
				std::swap(p[3], p[i]);
				check = true;
				break;
			}
		}
		if (!check) return false;
		for (int i = 0; i < (int)p.size(); i++)
			for (int j = 0; j < (int)p.size(); j++) {
				whe[i][j] = -1;
			}
		return true;
	}

	void pushface(const int &a, const int &b, const int &c) {
		nFace++;
		tmp[nFace] = Face(a, b, c);
		tmp[nFace].isOnConvex = true;
		whe[a][b] = nFace;
		whe[b][c] = nFace;
		whe[c][a] = nFace;
	}

	bool deal(const std::vector<TPoint> &p, const std::pair<int, int> &now, const TPoint &base) {
		int id = whe[now.second][now.first];
		if (!tmp[id].isOnConvex) return true;
		if (isVisible(p, tmp[id], base)) {
			queue[++right] = tmp[id];
			tmp[id].isOnConvex = false;
			return true;
		}
		return false;
	}

	std::vector<Triangle> getConvex(std::vector<TPoint> &p) {
		static std::vector<Triangle> ret;
		ret.clear();
		if (!init(p)) return ret;
		if (!isVisible(p, Face(0, 1, 2), p[3])) pushface(0, 1, 2); else pushface(0, 2, 1);
		if (!isVisible(p, Face(0, 1, 3), p[2])) pushface(0, 1, 3); else pushface(0, 3, 1);
		if (!isVisible(p, Face(0, 2, 3), p[1])) pushface(0, 2, 3); else pushface(0, 3, 2);
		if (!isVisible(p, Face(1, 2, 3), p[0])) pushface(1, 2, 3); else pushface(1, 3, 2);
		for (int a = 4; a < (int)p.size(); a++) {
			TPoint base = p[a];
			for (int i = 1; i <= nFace; i++) {
				if (tmp[i].isOnConvex && isVisible(p, tmp[i], base)) {
					left = 0, right = 0;
					queue[++right] = tmp[i];
					tmp[i].isOnConvex = false;
					while (left < right) {
						Face now = queue[++left];
						if (!deal(p, std::make_pair(now.a, now.b), base)) pushface(now.a, now.b, a);
						if (!deal(p, std::make_pair(now.b, now.c), base)) pushface(now.b, now.c, a);
						if (!deal(p, std::make_pair(now.c, now.a), base)) pushface(now.c, now.a, a);
					}
					break;
				}
			}
		}
		for (int i = 1; i <= nFace; i++) {
			Face now = tmp[i];
			if (now.isOnConvex) {
				ret.push_back(Triangle(p[now.a], p[now.b], p[now.c]));
			}
		}
		return ret;
	}
};

int n;
std::vector<TPoint> p;
std::vector<Triangle> answer;

int main() {
	scanf("%d", &n);
	for (int i = 1; i <= n; i++) {
		TPoint a;
		a.read();
		p.push_back(a);
	}
	answer = Convex_Hull::getConvex(p);
	double areaCounter = 0.0;
	for (int i = 0; i < (int)answer.size(); i++) {
		areaCounter += answer[i].getArea();
	}
	printf("%.3f\n", areaCounter);
	return 0;
}
\end{lstlisting}

			%\subsubsection{绕轴旋转}
		\section{多边形}
			\subsection{判断点在多边形内部}
				\begin{lstlisting}
bool point_on_line(const Point &p, const Point &a, const Point &b) {
    return sgn(det(p, a, b)) == 0 && sgn(dot(p, a, b)) <= 0;
}
bool point_in_polygon(const Point &p, const std::vector<Point> &polygon) {
    int counter = 0;
    for (int i = 0; i < (int)polygon.size(); ++i) {
        Point a = polygon[i], b = polygon[(i + 1) % (int)polygon.size()];
        if (point_on_line(p, a, b)) {
            //    Point on the boundary are excluded.
            return false;
        }
        int x = sgn(det(a, p, b));
        int y = sgn(a.y - p.y);
        int z = sgn(b.y - p.y);
        counter += (x > 0 && y <= 0 && z > 0);
        counter -= (x < 0 && z <= 0 && y > 0);
    }
    return counter;
}
\end{lstlisting}

			\subsection{多边形内整点计数}
				\begin{lstlisting}
int getInside(int n, Point *P) {  // 求多边形P内有多少个整数点
	int OnEdge = n;
	double area = getArea(n, P);
	for (int i = 0; i < n - 1; i++) {
		Point now = P[i + 1] - P[i];
		int y = (int)now.y, x = (int)now.x;
		OnEdge += abs(gcd(x, y)) - 1;
	}
	Point now = P[0] - P[n - 1];
	int y = (int)now.y, x = (int)now.x;
	OnEdge += abs(gcd(x, y)) - 1;
	double ret = area - (double)OnEdge / 2 + 1;
	return (int)ret;
}
\end{lstlisting}

			%\subsection{旋转卡壳}
			%\subsection{动态凸包}
			%\subsection{点到凸包的切线}
			%\subsection{直线与凸包的交点}
			%\subsection{凸多边形的交集}
			%\subsection{凸多边形内的最大圆}
		\section{圆}
			%\subsubsection{圆类}
			%\subsection{圆的交集}
			\subsection{最小覆盖圆}
				\begin{lstlisting}
Point getmid(Point a,Point b) {
	return Point((a.x + b.x) / 2, (a.y + b.y) / 2);
}
Point getcross(Point a, Point vA, Point b, Point vB) {
	Point u = a - b;
	double t = det(vB, u) / det(vA, vB);
	return a + vA * t;
}
Point getcir(Point a,Point b,Point c) {
	Point midA = getmid(a,b), vA = Point(-(b - a).y, (b - a).x);
	Point midB = getmid(b,c), vB = Point(-(c - b).y, (c - b).x);
	return getcross(midA, vA, midB, vB);
}
double mincir(Point *p,int n) {
	std::random_shuffle(p + 1, p + n + 1);
	Point O = p[1];
	double r = 0;
	for (int i = 2; i <= n; i++) {
		if (dist(O, p[i]) <= r) continue;
		O = p[i]; r = 0;
		for (int j = 1; j < i; j++) {
			if (dist(O, p[j]) <= r) continue;
			O = getmid(p[i], p[j]); r = dist(O,p[i]);
			for (int k = 1; k < j; k++) {
				if (dist(O,p[k]) <= r) continue;
				O = getcir(p[i], p[j], p[k]);
				r = dist(O,p[i]);
			}
		}
	}
	return r;
}
\end{lstlisting}

			%\subsubsection{最小覆盖球}
			%\subsubsection{判断圆存在交集}
			\subsection{多边形与圆的交面积}
				\begin{lstlisting}
// 求扇形面积
double getSectorArea(const Point &a, const Point &b, const double &r) {
	double c = (2.0 * r * r - sqrdist(a, b)) / (2.0 * r * r);
	double alpha = acos(c);
	return r * r * alpha / 2.0;
}
// 求二次方程ax^2 + bx + c = 0的解
std::pair<double, double> getSolution(const double &a, const double &b, const double &c) {
	double delta = b * b - 4.0 * a * c;
	if (dcmp(delta) < 0) return std::make_pair(0, 0);
	else return std::make_pair((-b - sqrt(delta)) / (2.0 * a), (-b + sqrt(delta)) / (2.0 * a));
}
// 直线与圆的交点
std::pair<Point, Point> getIntersection(const Point &a, const Point &b, const double &r) {
	Point d = b - a;
	double A = dot(d, d);
	double B = 2.0 * dot(d, a);
	double C = dot(a, a) - r * r;
	std::pair<double, double> s = getSolution(A, B, C);
	return std::make_pair(a + d * s.first, a + d * s.second);
}
// 原点到线段AB的距离
double getPointDist(const Point &a, const Point &b) {
	Point d = b - a;
	int sA = dcmp(dot(a, d)), sB = dcmp(dot(b, d));
	if (sA * sB <= 0) return det(a, b) / dist(a, b);
	else return std::min(dist(a), dist(b));
}
// a和b和原点组成的三角形与半径为r的圆的交的面积
double getArea(const Point &a, const Point &b, const double &r) {
	double dA = dot(a, a), dB = dot(b, b), dC = getPointDist(a, b), ans = 0.0;
	if (dcmp(dA - r * r) <= 0 && dcmp(dB - r * r) <= 0) return det(a, b) / 2.0;
	Point tA = a / dist(a) * r;
	Point tB = b / dist(b) * r;
	if (dcmp(dC - r) > 0) return getSectorArea(tA, tB, r);
	std::pair<Point, Point> ret = getIntersection(a, b, r);
	if (dcmp(dA - r * r) > 0 && dcmp(dB - r * r) > 0) {
		ans += getSectorArea(tA, ret.first, r);
		ans += det(ret.first, ret.second) / 2.0;
		ans += getSectorArea(ret.second, tB, r);
		return ans;
	}
	if (dcmp(dA - r * r) > 0) return det(ret.first, b) / 2.0 + getSectorArea(tA, ret.first, r);
	else return det(a, ret.second) / 2.0 + getSectorArea(ret.second, tB, r);
}
// 求圆与多边形的交的主过程
double getArea(int n, Point *p, const Point &c, const double r)  {
	double ret = 0.0;
	for (int i = 0; i < n; i++) {
		int sgn = dcmp(det(p[i] - c, p[(i + 1) % n] - c));
		if (sgn > 0) ret += getArea(p[i] - c, p[(i + 1) % n] - c, r);
		else ret -= getArea(p[(i + 1) % n] - c, p[i] - c, r);
	}
	return fabs(ret);
}
\end{lstlisting}

		%\subsection{三角形}
			%\subsubsection{三角形的内心}
			%\subsubsection{三角形的外心}
			%\subsubsection{三角形的垂心}
		%\subsection{黑暗科技}
			%\subsubsection{平面图形的转动惯量}
			%\subsubsection{平面区域处理}
			%\subsubsection{Vonoroi图}
	\chapter{其它}
		\section{STL使用方法}
			\subsection{nth\_element}
	用法:nth\_element(a + 1, a + id, a + n + 1); \par
	作用:将排名为$id$的元素放在第$id$个位置。
\subsection{next\_permutation}
	用法:next\_permutation(a + 1, a + n + 1); \par
	作用:以a中从小到大排序后为第一个排列,求得当期数组a中的下一个排列,返回值为当期排列是否为最后一个排列。

	\chapter{数学公式}
		\section{常用数学公式}
	\subsection{求和公式}
		\begin{enumerate}\setlength{\itemsep}{-\itemsep}
			\item $\sum_{k=1}^{n}(2k-1)^2 = \frac{n(4n^2-1)}{3}	$
			\item $\sum_{k=1}^{n}k^3 = [\frac{n(n+1)}{2}]^2	$
			\item $\sum_{k=1}^{n}(2k-1)^3 = n^2(2n^2-1)	$
			\item $\sum_{k=1}^{n}k^4 = \frac{n(n+1)(2n+1)(3n^2+3n-1)}{30}  $
			\item $\sum_{k=1}^{n}k^5 = \frac{n^2(n+1)^2(2n^2+2n-1)}{12}	$
			\item $\sum_{k=1}^{n}k(k+1) = \frac{n(n+1)(n+2)}{3}	$
			\item $\sum_{k=1}^{n}k(k+1)(k+2) = \frac{n(n+1)(n+2)(n+3)}{4} $
			\item $\sum_{k=1}^{n}k(k+1)(k+2)(k+3) = \frac{n(n+1)(n+2)(n+3)(n+4)}{5} $
		\end{enumerate}
	\subsection{斐波那契数列}
		\begin{enumerate}\setlength{\itemsep}{-\itemsep}
			\item $fib_0=0, fib_1=1, fib_n=fib_{n-1}+fib_{n-2}$
			\item $fib_{n+2} \cdot fib_n-fib_{n+1}^2=(-1)^{n+1}$
			\item $fib_{-n}=(-1)^{n-1}fib_n$
			\item $fib_{n+k}=fib_k \cdot fib_{n+1}+fib_{k-1} \cdot fib_n$
			\item $gcd(fib_m, fib_n)=fib_{gcd(m, n)}$
			\item $fib_m|fib_n^2\Leftrightarrow nfib_n|m$
		\end{enumerate}
	\subsection{错排公式}
		\begin{enumerate}\setlength{\itemsep}{-\itemsep}
			\item $D_n = (n-1)(D_{n-2}-D_{n-1})$
			\item $D_n = n! \cdot (1-\frac{1}{1!}+\frac{1}{2!}-\frac{1}{3!}+\ldots+\frac{(-1)^n}{n!})$
		\end{enumerate}
	\subsection{莫比乌斯函数}
		$\mu(n) = \begin{cases}
			1 & \text{若}n=1\\
			(-1)^k & \text{若}n\text{无平方数因子,且}n = p_1p_2\dots p_k\\
			0 & \text{若}n\text{有大于}1\text{的平方数因数}
		\end{cases}$
		\par
		$\sum_{d|n}{\mu(d)} = \begin{cases}
			1 & \text{若}n=1\\
			0 & \text{其他情况}
		\end{cases}$
		\par
		$g(n) = \sum_{d|n}{f(d)} \Leftrightarrow f(n) = \sum_{d|n}{\mu(d)g(\frac{n}{d})}$\par
		$g(x) = \sum_{n=1}^{[x]}f(\frac{x}{n}) \Leftrightarrow f(x) = \sum_{n=1}^{[x]}{\mu(n)g(\frac{x}{n})}$
	\subsection{Burnside引理}
		设$G$是一个有限群,作用在集合$X$上。对每个$g$属于$G$,令$X^g$表示$X$中在$g$作用下的不动元素,轨道数(记作$|X/G|$)由如下公式给出:
			$|X/G| = \frac{1}{|G|}\sum_{g \in G}|X^g|.\,$
	\subsection{五边形数定理}
		设$p(n)$是$n$的拆分数,有$p(n) = \sum_{k \in \mathbb{Z} \setminus \{0\}} (-1)^{k - 1} p\left(n - \frac{k(3k - 1)}{2}\right)$
	\subsection{树的计数}
		\begin{enumerate}\setlength{\itemsep}{-\itemsep}
			\item 有根树计数:$n+1$个结点的有根树的个数为
				$a_{n+1} = \frac{\sum_{j=1}^{n}{j \cdot a_j \cdot{S_{n, j}}}}{n}$
			其中,
				$S_{n, j} = \sum_{i=1}^{n/j}{a_{n+1-ij}} = S_{n-j, j} + a_{n+1-j}$
			\item 无根树计数:当$n$为奇数时,$n$个结点的无根树的个数为
				$a_n-\sum_{i=1}^{n/2}{a_ia_{n-i}}$
			当$n$为偶数时,$n$个结点的无根树的个数为
				$a_n-\sum_{i=1}^{n/2}{a_ia_{n-i}}+\frac{1}{2}a_{\frac{n}{2}}(a_{\frac{n}{2}}+1)$
			\item $n$个结点的完全图的生成树个数为
				$n^{n-2}$
			\item 矩阵-树定理:
			图$G$由$n$个结点构成,设$\bm{A}[G]$为图$G$的邻接矩阵、$\bm{D}[G]$为图$G$的度数矩阵,
			则图$G$的不同生成树的个数为$\bm{C}[G] = \bm{D}[G] - \bm{A}[G]$的任意一个$n-1$阶主子式的行列式值。
		\end{enumerate}
	\subsection{欧拉公式}
		平面图的顶点个数、边数和面的个数有如下关系:
			$V - E + F = C+ 1$
		\indent 其中,$V$是顶点的数目,$E$是边的数目,$F$是面的数目,$C$是组成图形的连通部分的数目。当图是单连通图的时候,公式简化为:
			$V - E + F = 2$
	\subsection{皮克定理}
		给定顶点坐标均是整点(或正方形格点)的简单多边形,其面积$A$和内部格点数目$i$、边上格点数目$b$的关系:
			$A = i + \frac{b}{2} - 1$
	\subsection{牛顿恒等式}
		设$\prod_{i = 1}^n{(x - x_i)} = a_n + a_{n - 1} x + \dots + a_1 x^{n - 1} + a_0 x^n$
		$p_k = \sum_{i = 1}^n{x_i^k}$
		则$a_0 p_k + a_1 p_{k - 1} + \cdots + a_{k - 1} p_1 + k a_k = 0$\par
		特别地,对于$|\bm{A} - \lambda \bm{E}| = (-1)^n(a_n + a_{n - 1} \lambda + \cdots + a_1 \lambda^{n - 1} + a_0 \lambda^n)$
		有$p_k = Tr(\bm{A}^k)$
	%\section{数论公式}
\section{平面几何公式}
	\subsection{三角形}
		\begin{enumerate}\setlength{\itemsep}{-\itemsep}
			\item 半周长
				$p=\frac{a+b+c}{2}$
			\item 面积
				$S=\frac{a \cdot H_a}{2}=\frac{ab \cdot sinC}{2}=\sqrt{p(p-a)(p-b)(p-c)}$
			\item 中线
				$M_a=\frac{\sqrt{2(b^2+c^2)-a^2}}{2}=\frac{\sqrt{b^2+c^2+2bc \cdot cosA}}{2}$
			\item 角平分线 
				$T_a=\frac{\sqrt{bc \cdot [(b+c)^2-a^2]}}{b+c}=\frac{2bc}{b+c}cos\frac{A}{2}$
			\item 高线
				$H_a=bsinC=csinB=\sqrt{b^2-(\frac{a^2+b^2-c^2}{2a})^2}$
			\item 内切圆半径
				\begin{align*}
					r&=\frac{S}{p}=\frac{arcsin\frac{B}{2} \cdot sin\frac{C}{2}}{sin\frac{B+C}{2}}=4R \cdot sin\frac{A}{2}sin\frac{B}{2}sin\frac{C}{2}\\
					&=\sqrt{\frac{(p-a)(p-b)(p-c)}{p}}=p \cdot tan\frac{A}{2}tan\frac{B}{2}tan\frac{C}{2}
				\end{align*}
			\item 外接圆半径
				$R=\frac{abc}{4S}=\frac{a}{2sinA}=\frac{b}{2sinB}=\frac{c}{2sinC}$
		\end{enumerate}
	\subsection{四边形}
		$D_1, D_2$为对角线,$M$对角线中点连线,$A$为对角线夹角,$p$为半周长
		\begin{enumerate}\setlength{\itemsep}{-\itemsep}
			\item $a^2+b^2+c^2+d^2=D_1^2+D_2^2+4M^2$
			\item $S=\frac{1}{2}D_1D_2sinA$
			\item 对于圆内接四边形
				$ac+bd=D_1D_2$
			\item 对于圆内接四边形
				$S=\sqrt{(p-a)(p-b)(p-c)(p-d)}$
		\end{enumerate}
	\subsection{正$n$边形}
		$R$为外接圆半径,$r$为内切圆半径
		\begin{enumerate}\setlength{\itemsep}{-\itemsep}
			\item 中心角
				$A=\frac{2\pi}{n}$
			\item 内角
				$C=\frac{n-2}{n}\pi$
			\item 边长
				$a=2\sqrt{R^2-r^2}=2R \cdot sin\frac{A}{2}=2r \cdot tan\frac{A}{2}$
			\item 面积
				$S=\frac{nar}{2}=nr^2 \cdot tan\frac{A}{2}=\frac{nR^2}{2} \cdot sinA=\frac{na^2}{4 \cdot tan\frac{A}{2}}$
		\end{enumerate}
	\subsection{圆}
		\begin{enumerate}\setlength{\itemsep}{-\itemsep}
			\item 弧长
				$l=rA$
			\item 弦长
				$a=2\sqrt{2hr-h^2}=2r\cdot sin\frac{A}{2}$
			\item 弓形高
				$h=r-\sqrt{r^2-\frac{a^2}{4}}=r(1-cos\frac{A}{2})=\frac{1}{2} \cdot arctan\frac{A}{4}$
			\item 扇形面积
				$S_1=\frac{rl}{2}=\frac{r^2A}{2}$
			\item 弓形面积
				$S_2=\frac{rl-a(r-h)}{2}=\frac{r^2}{2}(A-sinA)$
		\end{enumerate}
	\subsection{棱柱}
		\begin{enumerate}\setlength{\itemsep}{-\itemsep}
			\item 体积
				$V=Ah$
				$A$为底面积,$h$为高
			\item 侧面积
				$S=lp$
				$l$为棱长,$p$为直截面周长
			\item 全面积
				$T=S+2A$
		\end{enumerate}
	\subsection{棱锥}
		\begin{enumerate}\setlength{\itemsep}{-\itemsep}
			\item 体积
				$V=Ah$
				$A$为底面积,$h$为高
			\item 正棱锥侧面积
				$S=lp$
				$l$为棱长,$p$为直截面周长
			\item 正棱锥全面积
				$T=S+2A$
		\end{enumerate}
	\subsection{棱台}
		\begin{enumerate}\setlength{\itemsep}{-\itemsep}
			\item 体积
				$V=(A_1+A_2+\sqrt{A_1A_2}) \cdot \frac{h}{3}$
				$A_1,A_2$为上下底面积,$h$为高
			\item 正棱台侧面积
				$S=\frac{p_1+p_2}{2}l$
				$p_1,p_2$为上下底面周长,$l$为斜高
			\item 正棱台全面积
				$T=S+A_1+A_2$
		\end{enumerate}
	\subsection{圆柱}
		\begin{enumerate}\setlength{\itemsep}{-\itemsep}
			\item 侧面积
				$S=2\pi rh$
			\item 全面积
				$T=2\pi r(h+r)$
			\item 体积
				$V=\pi r^2h$
		\end{enumerate}
	\subsection{圆锥}
		\begin{enumerate}\setlength{\itemsep}{-\itemsep}
			\item 母线
				$l=\sqrt{h^2+r^2}$
			\item 侧面积
				$S=\pi rl$
			\item 全面积
				$T=\pi r(l+r)$
			\item 体积
				$V=\frac{\pi}{3} r^2h$
		\end{enumerate}
	\subsection{圆台}
		\begin{enumerate}\setlength{\itemsep}{-\itemsep}
			\item 母线
				$l=\sqrt{h^2+(r_1-r_2)^2}$
			\item 侧面积
				$S=\pi(r_1+r_2)l$
			\item 全面积
				$T=\pi r_1(l+r_1)+\pi r_2(l+r_2)$
			\item 体积
				$V=\frac{\pi}{3}(r_1^2+r_2^2+r_1r_2)h$
		\end{enumerate}
	\subsection{球}
		\begin{enumerate}\setlength{\itemsep}{-\itemsep}
			\item 全面积
				$T=4\pi r^2$
			\item 体积
				$V=\frac{4}{3}\pi r^3$
		\end{enumerate}
	\subsection{球台}
		\begin{enumerate}\setlength{\itemsep}{-\itemsep}
			\item 侧面积
				$S=2\pi rh$
			\item 全面积
				$T=\pi(2rh+r_1^2+r_2^2)$
			\item 体积
				$V=\frac{\pi h[3(r_1^2+r_2^2)+h^2]}{6}$
		\end{enumerate}
	\subsection{球扇形}
		\begin{enumerate}\setlength{\itemsep}{-\itemsep}
			\item 全面积
				$T=\pi r(2h+r_0)$
				$h$为球冠高,$r_0$为球冠底面半径
			\item 体积
				$V=\frac{2}{3}\pi r^2h$
		\end{enumerate}
\section{立体几何公式}
	\subsection{球面三角公式}
		设$a, b, c$是边长,$A, B, C$是所对的二面角,
		有余弦定理$cos a = cos b \cdot cos c + sin b \cdot sin c \cdot cos A$
		正弦定理$\frac{sin A}{sin a} = \frac{sin B}{sin b} = \frac{sin C}{sin c}$
		三角形面积是$A + B + C - \pi$
	\subsection{四面体体积公式}
		$U, V, W, u, v, w$是四面体的$6$条棱,$U, V, W$构成三角形,$(U, u), (V, v), (W, w)$互为对棱,
		则$V = \frac{\sqrt{(s - 2a)(s - 2b)(s - 2c)(s - 2d)}}{192 uvw}$
		其中$\left\{\begin{array}{lll}
				a & = & \sqrt{xYZ}, \\
				b & = & \sqrt{yZX}, \\
				c & = & \sqrt{zXY}, \\
				d & = & \sqrt{xyz}, \\
				s & = & a + b + c + d, \\ 
				X & = & (w - U + v)(U + v + w), \\
				x & = & (U - v + w)(v - w + U), \\
				Y & = & (u - V + w)(V + w + u), \\
				y & = & (V - w + u)(w - u + V), \\
				Z & = & (v - W + u)(W + u + v), \\
				z & = & (W - u + v)(u - v + W)
			\end{array}\right.$
\section{积分表}
\newcommand{\ud}{\mathrm{d}}
$\arcsin x \to \frac{1}{\sqrt{1-x^2}}				   $\par
$\arccos x \to -\frac{1}{\sqrt{1-x^2}}				  $\par
$\arctan x \to \frac{1}{1+x^2}						  $\par
$a^x \to \frac{a^x}{\ln a}							  $\par
$\sin x \to -\cos x									 $\par
$\cos x \to \sin x									  $\par
$\tan x \to -\ln\cos x								  $\par
$\sec x \to \ln\tan(\frac{x}{2}+\frac{\pi}{4})		  $\par
$\tan^2 x \to \tan x - x								$\par
$\csc x \to \ln\tan\frac{x}{2}						  $\par
$\sin^2 x \to \frac{x}{2} - \frac{1}{2}\sin x\cos x	 $\par
$\cos^2 x \to \frac{x}{2} + \frac{1}{2}\sin x\cos x	 $\par
$\sec^2 x \to \tan x									$\par
$\frac{1}{\sqrt{a^2-x^2}} \to \arcsin\frac{x}{a}		$\par
$\csc^2 x \to -\cot x								   $\par
$\frac{1}{a^2-x^2}(|x|<|a|) \to \frac{1}{2a}\ln\frac{a+x}{a-x}  $\par
$\frac{1}{x^2-a^2}(|x|>|a|) \to \frac{1}{2a}\ln\frac{x-a}{x+a}  $\par
$\sqrt{a^2-x^2} \to \frac{x}{2}\sqrt{a^2-x^2}+\frac{a^2}{2}\arcsin\frac{x}{a}   $\par
$\frac{1}{\sqrt{x^2+a^2}} \to \ln(x+\sqrt{a^2+x^2}) $\par
$\sqrt{a^2+x^2} \to \frac{x}{2}\sqrt{a^2+x^2}+\frac{a^2}{2}\ln(x+\sqrt{a^2+x^2})$\par
$\frac{1}{\sqrt{x^2-a^2}} \to \ln(x+\sqrt{x^2-a^2})$\par
$\sqrt{x^2-a^2} \to \frac{x}{2}\sqrt{x^2-a^2}-\frac{a^2}{2}\ln(x+\sqrt{x^2-a^2})$\par
$\frac{1}{x\sqrt{a^2-x^2}} \to -\frac{1}{a}\ln\frac{a+\sqrt{a^2-x^2}}{x}$\par
$\frac{1}{x\sqrt{x^2-a^2}} \to \frac{1}{a}\arccos\frac{a}{x}$\par
$\frac{1}{x\sqrt{a^2+x^2}} \to -\frac{1}{a}\ln\frac{a+\sqrt{a^2+x^2}}{x}$\par
$\frac{1}{\sqrt{2ax-x^2}} \to \arccos(1-\frac{x}{a})$\par
$\frac{x}{ax+b} \to \frac{x}{a}-\frac{b}{a^2}\ln(ax+b)$\par
$\sqrt{2ax-x^2} \to \frac{x-a}{2}\sqrt{2ax-x^2}+\frac{a^2}{2}\arcsin(\frac{x}{a}-1)$\par
$\frac{1}{x\sqrt{ax+b}}(b<0) \to \frac{2}{\sqrt{-b}}\arctan\sqrt{\frac{ax+b}{-b}}$\par
$x\sqrt{ax+b} \to \frac{2(3ax-2b)}{15a^2}(ax+b)^{\frac{3}{2}}$\par
$\frac{1}{x\sqrt{ax+b}}(b>0) \to \frac{1}{\sqrt{b}}\ln\frac{\sqrt{ax+b}-\sqrt{b}}{\sqrt{ax+b}+\sqrt{b}}$\par
$\frac{x}{\sqrt{ax+b}} \to \frac{2(ax-2b)}{3a^2}\sqrt{ax+b}$\par
$\frac{1}{x^2 \sqrt{ax+b}} \to -\frac{\sqrt{ax+b}}{bx}-\frac{a}{2b}\int\frac{\ud x}{x\sqrt{ax+b}}$\par
$\frac{\sqrt{ax+b}}{x} \to 2\sqrt{ax+b}+b\int\frac{\ud x}{x\sqrt{ax+b}}$\par
$\frac{1}{\sqrt{(ax+b)^n}}(n>2) \to \frac{-2}{a(n-2)}\cdot\frac{1}{\sqrt{(ax+b)^{n-2} }}$\par
$\frac{1}{ax^2+c}(a>0,c>0) \to \frac{1}{\sqrt{ac}}\arctan{(x\sqrt{\frac{a}{c}})}$\par
$\frac{x}{ax^2+c} \to \frac{1}{2a}\ln(ax^2+c)$\par
$\frac{1}{ax^2+c}(a+,c-) \to \frac{1}{2\sqrt{-ac}}\ln\frac{x\sqrt{a}-\sqrt{-c}}{x\sqrt{a}+\sqrt{-c}}$\par
$\frac{1}{x(ax^2+c)} \to \frac{1}{2c}\ln\frac{x^2}{ax^2+c}$\par
$\frac{1}{ax^2+c}(a-,c+) \to \frac{1}{2\sqrt{-ac}}\ln\frac{\sqrt{c}+x\sqrt{-a}}{\sqrt{c}-x\sqrt{-a}}$\par
$x{\sqrt{ax^2+c}} \to \frac{1}{3a}\sqrt{(ax^2+c)^3}$\par
$\frac{1}{(ax^2+c)^n}(n>1) \to \frac{x}{2c(n-1)(ax^2+c)^{n-1}}+\frac{2n-3}{2c(n-1)}\int\frac{\ud x}{(ax^2+c)^{n-1}}$\par
$\frac{x^n}{ax^2+c}(n\ne 1)\to \frac{x^{n-1}}{a(n-1)}-\frac{c}{a}\int\frac{x^{n-2}}{ax^2+c}\ud x$\par
$\frac{1}{x^2(ax^2+c)} \to \frac{-1}{cx}-\frac{a}{c}\int\frac{\ud x}{ax^2+c}$\par
$\frac{1}{x^2(ax^2+c)^n}(n\ge 2) \to \frac{1}{c}\int\frac{\ud x}{x^2(ax^2+c)^{n-1}}-\frac{a}{c}\int\frac{\ud x}{(ax^2+c)^n}$\par
$\sqrt{ax^2+c}(a>0) \to \frac{x}{2}\sqrt{ax^2+c}+\frac{c}{2\sqrt{a}}\ln(x\sqrt{a}+\sqrt{ax^2+c})$\par
$\sqrt{ax^2+c}(a<0) \to \frac{x}{2}\sqrt{ax^2+c}+\frac{c}{2\sqrt{-a}}\arcsin(x\sqrt{\frac{-a}{c}})$\par
$\frac{1}{\sqrt{ax^2+c}}(a>0) \to \frac{1}{\sqrt{a}}\ln(x\sqrt{a}+\sqrt{ax^2+c})$\par
$\frac{1}{\sqrt{ax^2+c}}(a<0) \to \frac{1}{\sqrt{-a}}\arcsin(x\sqrt{-\frac{a}{c}})$\par
$\sin^2 ax \to \frac{x}{2}-\frac{1}{4a}\sin 2ax$\par
$\cos^2 ax \to \frac{x}{2}+\frac{1}{4a}\sin 2ax$\par
$\frac{1}{\sin ax} \to \frac{1}{a}\ln\tan\frac{ax}{2}$\par
$\frac{1}{\cos^2 ax} \to \frac{1}{a}\tan ax$\par
$\frac{1}{\cos ax} \to \frac{1}{a}\ln \tan(\frac{\pi}{4}+\frac{ax}{2})$\par
$\ln(ax)\to x\ln(ax)-x$\par
$\sin^3 ax \to \frac{-1}{a}\cos ax+\frac{1}{3a}\cos^3 ax$\par
$\cos^3 ax \to \frac{1}{a}\sin ax - \frac{1}{3a}\sin^3 ax$\par
$\frac{1}{\sin^2 ax}\to -\frac{1}{a}\cot ax$\par
$x\ln(ax)\to \frac{x^2}{2}\ln(ax)-\frac{x^2}{4}$\par
$\cos ax\to \frac{1}{a}\sin ax$\par
$x^2 e^{ax} \to \frac{e^{ax}}{a^3}(a^2x^2-2ax+2)$\par
$(\ln(ax))^2 \to x(\ln(ax))^2-2x\ln(ax)+2x$\par
$x^2\ln(ax) \to \frac{x^3}{3}\ln(ax)-\frac{x^3}{9}$\par
$x^n\ln(ax) \to \frac{x^{n+1}}{n+1}\ln(ax)-\frac{x^{n+1}}{(n+1)^2}$\par
$\sin(\ln ax) \to \frac{x}{2}[\sin(\ln ax) - \cos(\ln ax)]$\par
$\cos(\ln ax) \to \frac{x}{2}[\sin(\ln ax) + \cos(\ln ax)]$\par

	\end{spacing}
\end{document}
