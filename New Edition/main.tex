\documentclass[a4paper,openany]{book}
%\usepackage{ctex}
\usepackage{bm}
%\usepackage[fleqn]{amsmath}
\usepackage{harpoon}
\usepackage{fontspec}
\usepackage{listings}
\usepackage[left=1.5cm, right=1.5cm, top=1.5cm, bottom=1.5cm]{geometry}
\usepackage{setspace}
\usepackage{bm}
\usepackage{cmap}
\usepackage{cite}
\usepackage{float}
\usepackage{xeCJK}
\usepackage{amsthm}
\usepackage{amsmath}
\usepackage{amssymb}
\usepackage{setspace}
\usepackage{enumerate}
\usepackage{indentfirst}
\usepackage[cache=false]{minted}
\allowdisplaybreaks

%\setlength{\parindent}{0em}
%\setlength{\mathindent}{0pt}

\newfontfamily\Courier{Courier New}
\renewcommand{\theFancyVerbLine}{\rmfamily\scriptsize\arabic{FancyVerbLine}}
\newcommand{\cppcode}[1]{
    \inputminted[mathescape,
    			 tabsize=4,
    			 linenos,
    			 frame=single,
    			 framesep=2mm,
    			 breakaftergroup=true,
    			 breakautoindent=true,
    			 breakbytoken=true,
    			 breaklines=true
    ]{cpp}{#1}
}

\newcommand{\javacode}[1]{
    \inputminted[mathescape]{java}{#1}
}
\begin{document}
	\title{\textbf{\LARGE{Standard Code Library}}}
	\author{Shanghai Jiao Tong University}
	\date{October, 2015}
	\maketitle
	\tableofcontents
	\begin{spacing}{1.0}
	\chapter{数论算法}
		\section{快速数论变换}
			使用条件及注意事项:$mod$必须要是一个形如$a2^b + 1$的数,$prt$表示$mod$的原根。
			\cppcode{Source/Number-Theory/Number-Theory-Transform.cpp}
		\section{多项式求逆}
			使用条件及注意事项:求一个多项式在模意义下的逆元。
			\cppcode{Source/Number-Theory/Inverse-Polynomial.cpp}
		\section{中国剩余定理}
			使用条件及注意事项:模数可以不互质。
			\cppcode{Source/Number-Theory/Chinese-Remainder-Theorem.cpp}
		\section{Miller Rabin}
			\cppcode{Source/Number-Theory/Miller-Rabin.cpp}
		\section{Pollard Rho}
			\cppcode{Source/Number-Theory/Pollard-Rho.cpp}
		\section{坚固的逆元}
			\cppcode{Source/Number-Theory/Quick-Inverse.cpp}
		\section{直线下整点个数}
			\cppcode{Source/Number-Theory/Lattice-Count.cpp}
	\chapter{数值算法}
		\section{快速傅立叶变换}
			\cppcode{Source/Numerical-Algorithm/Fast-Fourier-Transform.cpp}
		\section{单纯形法求解线性规划}
			使用条件及注意事项:返回结果为$max\{c_{1 \times m} \cdot x_{m \times 1} \ | \ x_{m \times 1} \geq 0_{m \times 1}, a_{n \times m} \cdot x_{m \times 1} \leq b_{n \times 1}\}$
			\cppcode{Source/Numerical-Algorithm/Linear-Programming-Simplex.cpp}
		\section{自适应辛普森}
			\cppcode{Source/Numerical-Algorithm/Adaptive-Simpson.cpp}
	\chapter{数据结构}
		\section{Splay普通操作版}
			使用条件及注意事项:\par
			\begin{enumerate}
				\item 插入$x$数
				\item 删除$x$数(若有多个相同的数,因只删除一个)
				\item 查询$x$数的排名(若有多个相同的数,因输出最小的排名)
				\item 查询排名为$x$的数
				\item 求$x$的前驱(前驱定义为小于$x$,且最大的数)
				\item 求$x$的后继(后继定义为大于$x$,且最小的数)
			\end{enumerate}
			\cppcode{Source/Data-Structure/Splay-Normal.cpp}
		\section{Splay区间操作版}
			使用条件及注意事项:\par
			这是为NOI2005维修数列的代码,仅供区间操作用的splay参考。
			\cppcode{Source/Data-Structure/Splay-Interval.cpp}
		\section{坚固的Treap}
			使用条件及注意事项:题目来源UVA 12358
			\cppcode{Source/Data-Structure/Persistent-Treap}
		\section{k-d树}
			使用条件及注意事项:这是求$k$远点的代码,要求$k$近点的话把堆的比较函数改一改,把朝左儿子或者是右儿子
			的方向改一改。
			\cppcode{Source/Data-Structure/K-Dimensional-Tree.cpp}
		\section{树链剖分}
			\subsection{点操作版本}
				使用条件及注意事项:树上最大(非空)子段和,注意一条路径询问的时候信息统计的顺序。
				\cppcode{Source/Data-Structure/Heavy-Light-Decomposition-Point.cpp}
			\subsection{链操作版本}
				\cppcode{Source/Data-Structure/Heavy-Light-Decomposition-Chain.cpp}
		\section{Link-Cut-Tree}
			\cppcode{Source/Data-Structure/Link-Cut-Tree.cpp}
	\chapter{图论}
		\section{强连通分量}
			\cppcode{Source/Graph-Theory/Strongly-Connected-Components.cpp}
		\section{点双连通分量}
		\subsection{坚固的点双连通分量}
			\cppcode{Source/Graph-Theory/Super-Double-Connected-Component.cpp}
		\subsection{朴素的点双连通分量}
			\cppcode{Source/Graph-Theory/Normal-Double-Connected-Component.cpp}
		\section{2-SAT问题}
			\cppcode{Source/Graph-Theory/Two-Satisfiability.cpp}
		\section{二分图最大匹配}
			\subsection{Hungary算法}
				时间复杂度:$\mathcal{O}(V \cdot E)$
				\cppcode{Source/Graph-Theory/Maximum-Matching-Hungary.cpp}
			\subsection{Hopcroft Karp算法}
				时间复杂度:$\mathcal{O}(\sqrt{V} \cdot E)$
				\cppcode{Source/Graph-Theory/Maximum-Matching-Hopcroft-Karp.cpp}
		\section{二分图最大权匹配}
			时间复杂度:$\mathcal{O}(V^4)$
			\cppcode{Source/Graph-Theory/Maximum-Weight-Matching.cpp}
		\section{最大流}
			\subsection{Dinic}
				使用方法以及注意事项:$n$个点,$m$条边,$inf$为一个很大的值,源点$s$,汇点$t$,图中最大点的编号为$t$。\par
				\indent 邻接表:$p$数组记录节点,$nxt$数组指向下一个位置,$c$数组记录可增广量,$h$数组记录表头(初始全为-1)。\par
				\indent 时间复杂度:$\mathcal{O}(V^2 \cdot E)$
				\cppcode{Source/Graph-Theory/Maximum-Flow-Dinic.cpp}
			\subsection{ISAP}
				\indent 时间复杂度:$\mathcal{O}(V^2 \cdot E)$
				\cppcode{Source/Graph-Theory/Maximum-Flow-ISAP.cpp}
			\subsection{SAP}
				\indent 时间复杂度:$\mathcal{O}(V^2 \cdot E)$
				\cppcode{Source/Graph-Theory/Maximum-Flow-SAP.cpp}
		\section{上下界网络流}
			$B(u,v)$表示边$(u,v)$流量的下界,$C(u,v)$表示边$(u,v)$流量的上界,$F(u,v)$表示边$(u,v)$的流量。
			设$G(u,v) = F(u,v) - B(u,v)$,显然有
			$$0 \leq G(u,v) \leq C(u,v)-B(u,v)$$
		\subsection{无源汇的上下界可行流}
			建立超级源点$S^*$和超级汇点$T^*$,对于原图每条边$(u,v)$在新网络中连如下三条边:$S^* \rightarrow v$,容量为$B(u,v)$;$u \rightarrow T^*$,容量为$B(u,v)$;$u \rightarrow v$,容量为$C(u,v) - B(u,v)$。最后求新网络的最大流,判断从超级源点$S^*$出发的边是否都满流即可,边$(u,v)$的最终解中的实际流量为$G(u,v)+B(u,v)$。
		\subsection{有源汇的上下界可行流}
			从汇点$T$到源点$S$连一条上界为$\infty$,下界为$0$的边。按照\textbf{无源汇的上下界可行流}一样做即可,流量即为$T \rightarrow S$边上的流量。
		\subsection{有源汇的上下界最大流}
			\begin{enumerate}
				\item 在\textbf{有源汇的上下界可行流}中,从汇点$T$到源点$S$的边改为连一条上界为$\infty$,下届为$x$的边。$x$满足二分性质,找到最大的$x$使得新网络存在\textbf{无源汇的上下界可行流}即为原图的最大流。
				\item 从汇点$T$到源点$S$连一条上界为$\infty$,下界为$0$的边,变成无源汇的网络。按照\textbf{无源汇的上下界可行流}的方法,建立超级源点$S^*$和超级汇点$T^*$,求一遍$S^* \rightarrow T^*$的最大流,再将从汇点$T$到源点$S$的这条边拆掉,求一次$S \rightarrow T$的最大流即可。
			\end{enumerate}
		\subsection{有源汇的上下界最小流}
			\begin{enumerate}
				\item 在\textbf{有源汇的上下界可行流}中,从汇点$T$到源点$S$的边改为连一条上界为$x$,下界为$0$的边。$x$满足二分性质,找到最小的$x$使得新网络存在\textbf{无源汇的上下界可行流}即为原图的最小流。
				\item 按照\textbf{无源汇的上下界可行流}的方法,建立超级源点$S^*$与超级汇点$T^*$,求一遍$S^* \rightarrow T^*$的最大流,但是注意这一次不加上汇点$T$到源点$S$的这条边,即不使之改为无源汇的网络去求解。求完后,再加上那条汇点$T$到源点$S$上界$\infty$的边。因为这条边下界为$0$,所以$S^*$,$T^*$无影响,再直接求一次$S^* \rightarrow T^*$的最大流。若超级源点$S^*$出发的边全部满流,则$T \rightarrow S$边上的流量即为原图的最小流,否则无解。
			\end{enumerate}
		\section{最小费用最大流}
		\subsection{稀疏图}
			时间复杂度:$\mathcal{O}(V \cdot E^2)$
			\cppcode{Source/Graph-Theory/Minimum-Cost-Flow-Spfa.cpp}
		\subsection{稠密图}
			使用条件:费用非负\\
			\indent 时间复杂度:$\mathcal{O}(V \cdot E^2)$
			\cppcode{Source/Graph-Theory/Minimum-Cost-Flow-Zkw.cpp}
		\section{一般图最大匹配}
			时间复杂度:$\mathcal{O}(V^3)$
			\cppcode{Source/Graph-Theory/Maximum-Matching-Blossom.cpp}
		\section{无向图全局最小割}
			时间复杂度:$\mathcal{O}(V^3)$\\
			\indent 注意事项:处理重边时,应该对边权累加
			\cppcode{Source/Graph-Theory/Minimum-Cut-Stoer-Wagner.cpp}
		\section{最小树形图}
			\cppcode{Source/Graph-Theory/Chu-Liu-Algorithm.cpp}
		\section{有根树的同构}
			时间复杂度:$\mathcal{O}(V log V)$
			\cppcode{Source/Graph-Theory/Rooted-Tree-Isomorphism.cpp}
		\section{度限制生成树}
			\cppcode{Source/Graph-Theory/Minimum-Spanning-Tree-With-Degree-Limit.cpp}
		\section{弦图相关}
			\subsection{弦图的判定}
				\cppcode{Source/Graph-Theory/Chord-Graph-Judgement.cpp}
			\subsection{弦图的团数}
				\cppcode{Source/Graph-Theory/Chord-Graph-Group-Counter.cpp}
		\section{哈密尔顿回路(ORE性质的图)}
			ORE性质:$$\forall x,y \in V \wedge (x,y) \notin E \ \ s.t. \ \ deg_x+deg_y \geq n$$
			\indent 返回结果:从顶点$1$出发的一个哈密尔顿回路\\
			\indent 使用条件:$n \geq 3$
			\cppcode{Source/Graph-Theory/Hamiltonian-Circuit-Ore.cpp}
	\chapter{字符串}
		\section{模式串匹配}
			\cppcode{Source/String-Algorithm/Knuth-Morris-Pratt.cpp}
		\section{坚固的模式串匹配}
			\cppcode{Source/String-Algorithm/Extended-Knuth-Morris-Pratt.cpp}
		\section{AC自动机}
			\cppcode{Source/String-Algorithm/Aho-Corasick-Automaton.cpp}
		\section{后缀数组}
			\cppcode{Source/String-Algorithm/Suffix-Array.cpp}
		\section{广义后缀自动机}
			\cppcode{Source/String-Algorithm/Generalized-Suffix-Automaton.cpp}
		\section{Manacher算法}
			\cppcode{Source/String-Algorithm/Manacher.cpp}
		\section{回文树}
			\cppcode{Source/String-Algorithm/Palindromic-Tree.cpp}
		\section{循环串最小表示}
			\cppcode{Source/String-Algorithm/Minimum-Circular-Representation.cpp}
	\chapter{计算几何}
		\section{二维基础}
			\subsection{点类}
				\cppcode{Source/Computational-Geometry/Point-Class-2D.cpp}
			\subsection{凸包}
				\cppcode{Source/Computational-Geometry/Convex-Hull-2D.cpp}
			\subsection{半平面交}
				\cppcode{Source/Computational-Geometry/Half-Plane-Intersection.cpp}
			\subsection{最近点对}
				\cppcode{Source/Computational-Geometry/Closest-Pair-Of-Points.cpp}
		\section{三维基础}
			\subsection{点类}
				\cppcode{Source/Computational-Geometry/Point-Class-3D.cpp}
			\subsection{凸包}
				\cppcode{Source/Computational-Geometry/Convex-Hull-3D.cpp}
			\subsection{绕轴旋转}
				使用方法及注意事项:逆时针绕轴$AB$旋转$\theta$角
				\cppcode{Source/Computational-Geometry/Rotate-3D.cpp}
		\section{多边形}
			\subsection{判断点在多边形内部}
				\cppcode{Source/Computational-Geometry/Point-In-Polygon.cpp}
			\subsection{多边形内整点计数}
				\cppcode{Source/Computational-Geometry/Lattice-In-Polygon-Counter.cpp}
			%\subsection{旋转卡壳}
			%\subsection{动态凸包}
			%\subsection{点到凸包的切线}
			%\subsection{直线与凸包的交点}
			%\subsection{凸多边形的交集}
			%\subsection{凸多边形内的最大圆}
		\section{圆}
			%\subsubsection{圆类}
			%\subsection{圆的交集}
			\subsection{最小覆盖圆}
				\cppcode{Source/Computational-Geometry/Minimum-Coverage-Circle.cpp}
			\subsection{最小覆盖球}
				\cppcode{Source/Computational-Geometry/Minimum-Coverage-Ball.cpp}
			%\subsubsection{判断圆存在交集}
			\subsection{多边形与圆的交面积}
				\cppcode{Source/Computational-Geometry/Polygon-Circle-Intersection.cpp}
		%\subsection{三角形}
			%\subsubsection{三角形的内心}
			%\subsubsection{三角形的外心}
			%\subsubsection{三角形的垂心}
		%\subsection{黑暗科技}
			%\subsubsection{平面图形的转动惯量}
			%\subsubsection{平面区域处理}
			%\subsubsection{Vonoroi图}
	\chapter{其它}
		\section{STL使用方法}
			\subsection{nth\_element}
	用法:nth\_element(a + 1, a + id, a + n + 1); \par
	作用:将排名为$id$的元素放在第$id$个位置。
\subsection{next\_permutation}
	用法:next\_permutation(a + 1, a + n + 1); \par
	作用:以a中从小到大排序后为第一个排列,求得当期数组a中的下一个排列,返回值为当期排列是否为最后一个排列。

		\section{博弈论相关}
			\subsection{巴什博奕}
	\begin{enumerate}
		\item 
			只有一堆n个物品,两个人轮流从这堆物品中取物,规定每次至少取一个,最多取m个。最后取光者得胜。
		\item
			显然,如果$n=m+1$,那么由于一次最多只能取$m$个,所以,无论先取者拿走多少个,
			后取者都能够一次拿走剩余的物品,后者取胜。因此我们发现了如何取胜的法则:如果
			$n=(m+1)r+s$,(r为任意自然数,$s \leq m$),那么先取者要拿走$s$个物品,
			如果后取者拿走$k(k \leq m)$个,那么先取者再拿走$m+1-k$个,结果剩下$(m+1)(r-1)$
			个,以后保持这样的取法,那么先取者肯定获胜。总之,要保持给对手留下$(m+1)$的倍数,
			就能最后获胜。
	\end{enumerate}
\subsection{威佐夫博弈}
	\begin{enumerate}
		\item 
			有两堆各若干个物品,两个人轮流从某一堆或同时从两堆中取同样多的物品,规定每次至少取
			一个,多者不限,最后取光者得胜。
		\item
			判断一个局势$(a, b)$为奇异局势(必败态)的方法:
			$$a_k =[k (1+\sqrt{5})/2],b_k= a_k + k$$
	\end{enumerate}
\subsection{阶梯博奕}
	\begin{enumerate}
		\item
			博弈在一列阶梯上进行,每个阶梯上放着自然数个点,两个人进行阶梯博弈,
			每一步则是将一个阶梯上的若干个点(至少一个)移到前面去,最后没有点
			可以移动的人输。
		\item
			解决方法:把所有奇数阶梯看成N堆石子,做NIM。(把石子从奇数堆移动到偶数
			堆可以理解为拿走石子,就相当于几个奇数堆的石子在做Nim)
	\end{enumerate}
\subsection{图上删边游戏}
	\subsubsection{链的删边游戏}
		\begin{enumerate}
			\item
				游戏规则:对于一条链,其中一个端点是根,两人轮流删边,脱离根的部分也算被删去,最后没边可删的人输。
			\item
				做法:$sg[i] = n - dist(i) - 1$(其中$n$表示总点数,$dist(i)$表示离根的距离)
		\end{enumerate}
	\subsubsection{树的删边游戏}
		\begin{enumerate}
			\item
				游戏规则:对于一棵有根树,两人轮流删边,脱离根的部分也算被删去,没边可删的人输。
			\item
				做法:叶子结点的$sg=0$,其他节点的$sg$等于儿子结点的$sg+1$的异或和。
		\end{enumerate}
	\subsubsection{局部连通图的删边游戏}
		\begin{enumerate}
			\item
				游戏规则:在一个局部连通图上,两人轮流删边,脱离根的部分也算被删去,没边可删的人输。
				局部连通图的构图规则是,在一棵基础树上加边得到,所有形成的环保证不共用边,且只与基础树有一个公共点。
			\item
				做法:去掉所有的偶环,将所有的奇环变为长度为1的链,然后做树的删边游戏。
		\end{enumerate}

		\section{Java Reference}
			\lstset{
	language=JAVA,
	tabsize=4,
	numbers=left,
	breaklines=tr,
	extendedchars=false
	xleftmargin=0em,
	xrightmargin=0em,
	aboveskip=1em,
	numberstyle=\small\Courier,
    basicstyle=\small\Courier
}
\begin{lstlisting}
import java.io.*;
import java.util.*;
import java.math.*;

public class Main {
	static int get(char c) {
		if (c <= '9')
			return c - '0';
		else if (c <= 'Z')
			return c - 'A' + 10;
		else
			return c - 'a' + 36;
	}
	static char get(int x) {
		if (x <= 9)
			return (char)(x + '0');
		else if (x <= 35)
			return (char)(x - 10 + 'A');
		else
			return (char)(x - 36 + 'a');
	}
	static BigInteger get(String s, BigInteger x) {
		BigInteger ans = BigInteger.valueOf(0), now = BigInteger.valueOf(1);
		for (int i = s.length() - 1; i >= 0; i--) {
			ans = ans.add(now.multiply(BigInteger.valueOf(get(s.charAt(i)))));
			now = now.multiply(x);
		}
		return ans;
	}
	public static void main(String [] args) {
		Scanner cin = new Scanner(new BufferedInputStream(System.in));
		for (; ; ) {
			BigInteger x = cin.nextBigInteger();
			if (x.compareTo(BigInteger.valueOf(0)) == 0)
				break;
			String s = cin.next(), t = cin.next(), r = "";
			BigInteger ans = get(s, x).mod(get(t, x));
			if (ans.compareTo(BigInteger.valueOf(0)) == 0)
				r = "0";
			for (; ans.compareTo(BigInteger.valueOf(0)) > 0;) {
				r = get(ans.mod(x).intValue()) + r;
				ans = ans.divide(x);
			}
			System.out.println(r);
		}
	}
}

// Arrays
int a[];
.fill(a[, int fromIndex, int toIndex],val); | .sort(a[, int fromIndex, int toIndex])
// String
String s;
.charAt(int i); | compareTo(String) | compareToIgnoreCase () | contains(String) |
length () | substring(int l, int len)
// BigInteger
.abs() | .add() | bitLength () | subtract () | divide () | remainder () | divideAndRemainder () | modPow(b, c) |
pow(int) | multiply () | compareTo () |
gcd() | intValue () | longValue () | isProbablePrime(int c) (1 - 1/2^c) |
nextProbablePrime () | shiftLeft(int) | valueOf ()
// BigDecimal
.ROUND_CEILING | ROUND_DOWN_FLOOR | ROUND_HALF_DOWN | ROUND_HALF_EVEN | ROUND_HALF_UP | ROUND_UP
.divide(BigDecimal b, int scale , int round_mode) | doubleValue () | movePointLeft(int) | pow(int) |
setScale(int scale , int round_mode) | stripTrailingZeros ()
// StringBuilder
StringBuilder sb = new StringBuilder ();
sb.append(elem) | out.println(sb)
\end{lstlisting}

	\chapter{数学公式}
		\section{常用数学公式}
	\subsection{求和公式}
		\begin{enumerate}
			\item $\sum_{k=1}^{n}(2k-1)^2 = \frac{n(4n^2-1)}{3}	$
			\item $\sum_{k=1}^{n}k^3 = [\frac{n(n+1)}{2}]^2	$
			\item $\sum_{k=1}^{n}(2k-1)^3 = n^2(2n^2-1)	$
			\item $\sum_{k=1}^{n}k^4 = \frac{n(n+1)(2n+1)(3n^2+3n-1)}{30}  $
			\item $\sum_{k=1}^{n}k^5 = \frac{n^2(n+1)^2(2n^2+2n-1)}{12}	$
			\item $\sum_{k=1}^{n}k(k+1) = \frac{n(n+1)(n+2)}{3}	$
			\item $\sum_{k=1}^{n}k(k+1)(k+2) = \frac{n(n+1)(n+2)(n+3)}{4} $
			\item $\sum_{k=1}^{n}k(k+1)(k+2)(k+3) = \frac{n(n+1)(n+2)(n+3)(n+4)}{5} $
		\end{enumerate}
	\subsection{斐波那契数列}
		\begin{enumerate}
			\item $fib_0=0, fib_1=1, fib_n=fib_{n-1}+fib_{n-2}$
			\item $fib_{n+2} \cdot fib_n-fib_{n+1}^2=(-1)^{n+1}$
			\item $fib_{-n}=(-1)^{n-1}fib_n$
			\item $fib_{n+k}=fib_k \cdot fib_{n+1}+fib_{k-1} \cdot fib_n$
			\item $gcd(fib_m, fib_n)=fib_{gcd(m, n)}$
			\item $fib_m|fib_n^2\Leftrightarrow nfib_n|m$
		\end{enumerate}
	\subsection{错排公式}
		\begin{enumerate}
			\item $D_n = (n-1)(D_{n-2}-D_{n-1})$
			\item $D_n = n! \cdot (1-\frac{1}{1!}+\frac{1}{2!}-\frac{1}{3!}+\ldots+\frac{(-1)^n}{n!})$
		\end{enumerate}
	\subsection{莫比乌斯函数}
		$$\mu(n) = \begin{cases}
			1 & \text{若}n=1\\
			(-1)^k & \text{若}n\text{无平方数因子,且}n = p_1p_2\dots p_k\\
			0 & \text{若}n\text{有大于}1\text{的平方数因数}
		\end{cases}$$
		$$\sum_{d|n}{\mu(d)} = \begin{cases}
			1 & \text{若}n=1\\
			0 & \text{其他情况}
		\end{cases}$$
		$$g(n) = \sum_{d|n}{f(d)} \Leftrightarrow f(n) = \sum_{d|n}{\mu(d)g(\frac{n}{d})}$$
		$$g(x) = \sum_{n=1}^{[x]}f(\frac{x}{n}) \Leftrightarrow f(x) = \sum_{n=1}^{[x]}{\mu(n)g(\frac{x}{n})}$$
	\subsection{Burnside引理}
		设$G$是一个有限群,作用在集合$X$上。对每个$g$属于$G$,令$X^g$表示$X$中在$g$作用下的不动元素,轨道数(记作$|X/G|$)由如下公式给出:
			$$|X/G| = \frac{1}{|G|}\sum_{g \in G}|X^g|.\,$$
	\subsection{五边形数定理}
		设$p(n)$是$n$的拆分数,有$$p(n) = \sum_{k \in \mathbb{Z} \setminus \{0\}} (-1)^{k - 1} p\left(n - \frac{k(3k - 1)}{2}\right)$$
	\subsection{树的计数}
		\begin{enumerate}
			\item 有根树计数:$n+1$个结点的有根树的个数为
				$$a_{n+1} = \frac{\sum_{j=1}^{n}{j \cdot a_j \cdot{S_{n, j}}}}{n}$$
			其中,
				$$S_{n, j} = \sum_{i=1}^{n/j}{a_{n+1-ij}} = S_{n-j, j} + a_{n+1-j}$$
			\item 无根树计数:当$n$为奇数时,$n$个结点的无根树的个数为
				$$a_n-\sum_{i=1}^{n/2}{a_ia_{n-i}}$$
			当$n$为偶数时,$n$个结点的无根树的个数为
				$$a_n-\sum_{i=1}^{n/2}{a_ia_{n-i}}+\frac{1}{2}a_{\frac{n}{2}}(a_{\frac{n}{2}}+1)$$
			\item $n$个结点的完全图的生成树个数为
				$$n^{n-2}$$
			\item 矩阵-树定理:
			图$G$由$n$个结点构成,设$\bm{A}[G]$为图$G$的邻接矩阵、$\bm{D}[G]$为图$G$的度数矩阵,
			则图$G$的不同生成树的个数为$\bm{C}[G] = \bm{D}[G] - \bm{A}[G]$的任意一个$n-1$阶主子式的行列式值。
		\end{enumerate}
	\subsection{欧拉公式}
		平面图的顶点个数、边数和面的个数有如下关系:
			$$V - E + F = C+ 1$$
		\indent 其中,$V$是顶点的数目,$E$是边的数目,$F$是面的数目,$C$是组成图形的连通部分的数目。当图是单连通图的时候,公式简化为:
			$$V - E + F = 2$$
	\subsection{皮克定理}
		给定顶点坐标均是整点(或正方形格点)的简单多边形,其面积$A$和内部格点数目$i$、边上格点数目$b$的关系:
			$$A = i + \frac{b}{2} - 1$$
	\subsection{牛顿恒等式}
		设$$\prod_{i = 1}^n{(x - x_i)} = a_n + a_{n - 1} x + \dots + a_1 x^{n - 1} + a_0 x^n$$
		$$p_k = \sum_{i = 1}^n{x_i^k}$$
		则$$a_0 p_k + a_1 p_{k - 1} + \cdots + a_{k - 1} p_1 + k a_k = 0$$\par
		特别地,对于$$|\bm{A} - \lambda \bm{E}| = (-1)^n(a_n + a_{n - 1} \lambda + \cdots + a_1 \lambda^{n - 1} + a_0 \lambda^n)$$
		有$$p_k = Tr(\bm{A}^k)$$
	%\section{数论公式}
\section{平面几何公式}
	\subsection{三角形}
		\begin{enumerate}
			\item 半周长
				$$p=\frac{a+b+c}{2}$$
			\item 面积
				$$S=\frac{a \cdot H_a}{2}=\frac{ab \cdot sinC}{2}=\sqrt{p(p-a)(p-b)(p-c)}$$
			\item 中线
				$$M_a=\frac{\sqrt{2(b^2+c^2)-a^2}}{2}=\frac{\sqrt{b^2+c^2+2bc \cdot cosA}}{2}$$
			\item 角平分线 
				$$T_a=\frac{\sqrt{bc \cdot [(b+c)^2-a^2]}}{b+c}=\frac{2bc}{b+c}cos\frac{A}{2}$$
			\item 高线
				$$H_a=bsinC=csinB=\sqrt{b^2-(\frac{a^2+b^2-c^2}{2a})^2}$$
			\item 内切圆半径
				\begin{align*}
					r&=\frac{S}{p}=\frac{arcsin\frac{B}{2} \cdot sin\frac{C}{2}}{sin\frac{B+C}{2}}=4R \cdot sin\frac{A}{2}sin\frac{B}{2}sin\frac{C}{2}\\
					&=\sqrt{\frac{(p-a)(p-b)(p-c)}{p}}=p \cdot tan\frac{A}{2}tan\frac{B}{2}tan\frac{C}{2}
				\end{align*}
			\item 外接圆半径
				$$R=\frac{abc}{4S}=\frac{a}{2sinA}=\frac{b}{2sinB}=\frac{c}{2sinC}$$
		\end{enumerate}
	\subsection{四边形}
		$D_1, D_2$为对角线,$M$对角线中点连线,$A$为对角线夹角,$p$为半周长
		\begin{enumerate}
			\item $a^2+b^2+c^2+d^2=D_1^2+D_2^2+4M^2$
			\item $S=\frac{1}{2}D_1D_2sinA$
			\item 对于圆内接四边形
				$$ac+bd=D_1D_2$$
			\item 对于圆内接四边形
				$$S=\sqrt{(p-a)(p-b)(p-c)(p-d)}$$
		\end{enumerate}
	\subsection{正$n$边形}
		$R$为外接圆半径,$r$为内切圆半径
		\begin{enumerate}
			\item 中心角
				$$A=\frac{2\pi}{n}$$
			\item 内角
				$$C=\frac{n-2}{n}\pi$$
			\item 边长
				$$a=2\sqrt{R^2-r^2}=2R \cdot sin\frac{A}{2}=2r \cdot tan\frac{A}{2}$$
			\item 面积
				$$S=\frac{nar}{2}=nr^2 \cdot tan\frac{A}{2}=\frac{nR^2}{2} \cdot sinA=\frac{na^2}{4 \cdot tan\frac{A}{2}}$$
		\end{enumerate}
	\subsection{圆}
		\begin{enumerate}
			\item 弧长
				$$l=rA$$
			\item 弦长
				$$a=2\sqrt{2hr-h^2}=2r\cdot sin\frac{A}{2}$$
			\item 弓形高
				$$h=r-\sqrt{r^2-\frac{a^2}{4}}=r(1-cos\frac{A}{2})=\frac{1}{2} \cdot arctan\frac{A}{4}$$
			\item 扇形面积
				$$S_1=\frac{rl}{2}=\frac{r^2A}{2}$$
			\item 弓形面积
				$$S_2=\frac{rl-a(r-h)}{2}=\frac{r^2}{2}(A-sinA)$$
		\end{enumerate}
	\subsection{棱柱}
		\begin{enumerate}
			\item 体积
				$$V=Ah$$
				$A$为底面积,$h$为高
			\item 侧面积
				$$S=lp$$
				$l$为棱长,$p$为直截面周长
			\item 全面积
				$$T=S+2A$$
		\end{enumerate}
	\subsection{棱锥}
		\begin{enumerate}
			\item 体积
				$$V=Ah$$
				$A$为底面积,$h$为高
			\item 正棱锥侧面积
				$$S=lp$$
				$l$为棱长,$p$为直截面周长
			\item 正棱锥全面积
				$$T=S+2A$$
		\end{enumerate}
	\subsection{棱台}
		\begin{enumerate}
			\item 体积
				$$V=(A_1+A_2+\sqrt{A_1A_2}) \cdot \frac{h}{3}$$
				$A_1,A_2$为上下底面积,$h$为高
			\item 正棱台侧面积
				$$S=\frac{p_1+p_2}{2}l$$
				$p_1,p_2$为上下底面周长,$l$为斜高
			\item 正棱台全面积
				$$T=S+A_1+A_2$$
		\end{enumerate}
	\subsection{圆柱}
		\begin{enumerate}
			\item 侧面积
				$$S=2\pi rh$$
			\item 全面积
				$$T=2\pi r(h+r)$$
			\item 体积
				$$V=\pi r^2h$$
		\end{enumerate}
	\subsection{圆锥}
		\begin{enumerate}
			\item 母线
				$$l=\sqrt{h^2+r^2}$$
			\item 侧面积
				$$S=\pi rl$$
			\item 全面积
				$$T=\pi r(l+r)$$
			\item 体积
				$$V=\frac{\pi}{3} r^2h$$
		\end{enumerate}
	\subsection{圆台}
		\begin{enumerate}
			\item 母线
				$$l=\sqrt{h^2+(r_1-r_2)^2}$$
			\item 侧面积
				$$S=\pi(r_1+r_2)l$$
			\item 全面积
				$$T=\pi r_1(l+r_1)+\pi r_2(l+r_2)$$
			\item 体积
				$$V=\frac{\pi}{3}(r_1^2+r_2^2+r_1r_2)h$$
		\end{enumerate}
	\subsection{球}
		\begin{enumerate}
			\item 全面积
				$$T=4\pi r^2$$
			\item 体积
				$$V=\frac{4}{3}\pi r^3$$
		\end{enumerate}
	\subsection{球台}
		\begin{enumerate}
			\item 侧面积
				$$S=2\pi rh$$
			\item 全面积
				$$T=\pi(2rh+r_1^2+r_2^2)$$
			\item 体积
				$$V=\frac{\pi h[3(r_1^2+r_2^2)+h^2]}{6}$$
		\end{enumerate}
	\subsection{球扇形}
		\begin{enumerate}
			\item 全面积
				$$T=\pi r(2h+r_0)$$
				$h$为球冠高,$r_0$为球冠底面半径
			\item 体积
				$$V=\frac{2}{3}\pi r^2h$$
		\end{enumerate}
\section{立体几何公式}
	\subsection{球面三角公式}
		设$a, b, c$是边长,$A, B, C$是所对的二面角,
		有余弦定理$$cos a = cos b \cdot cos c + sin b \cdot sin c \cdot cos A$$
		正弦定理$$\frac{sin A}{sin a} = \frac{sin B}{sin b} = \frac{sin C}{sin c}$$
		三角形面积是$A + B + C - \pi$
	\subsection{四面体体积公式}
		$U, V, W, u, v, w$是四面体的$6$条棱,$U, V, W$构成三角形,$(U, u), (V, v), (W, w)$互为对棱,
		则$$V = \frac{\sqrt{(s - 2a)(s - 2b)(s - 2c)(s - 2d)}}{192 uvw}$$
		其中$$\left\{\begin{array}{lll}
				a & = & \sqrt{xYZ}, \\
				b & = & \sqrt{yZX}, \\
				c & = & \sqrt{zXY}, \\
				d & = & \sqrt{xyz}, \\
				s & = & a + b + c + d, \\ 
				X & = & (w - U + v)(U + v + w), \\
				x & = & (U - v + w)(v - w + U), \\
				Y & = & (u - V + w)(V + w + u), \\
				y & = & (V - w + u)(w - u + V), \\
				Z & = & (v - W + u)(W + u + v), \\
				z & = & (W - u + v)(u - v + W)
			\end{array}\right.$$

	\end{spacing}
\end{document}
