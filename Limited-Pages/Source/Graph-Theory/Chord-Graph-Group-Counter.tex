\begin{lstlisting}
int n, m, first[100001], next[2000001], where[2000001], l, L[100001], R[100001], c[100001], f[100001],
pos[100001], idx[100001], v[100001], ans;
inline void makelist(int x, int y){
    where[++l] = y;
    next[l] = first[x];
    first[x] = l;
}
int main(){
    memset(first, 0, sizeof(first)); l = 0;
    n = read(); m = read();
    for (int i = 1; i <= m; i++) {
        int x, y;
        x = read(); y = read();
        makelist(x, y); makelist(y, x);
    }
    memset(L, 0, sizeof(L));
    memset(R, 255, sizeof(R));
    memset(f, 0, sizeof(f));
    memset(idx, 0, sizeof(idx));
    for (int i = 1; i <= n; i++) c[i] = i, pos[i] = i;
    L[0] = 1; R[0] = n; ans = 0;
    for (int i = n; i; --i) {
        int now = c[i], cnt = 1;
        idx[now] = i; v[i] = now;
        if (--R[f[now]] < L[f[now]]) R[f[now]] = -1;
        for (int x = first[now]; x; x = next[x])
            if (!idx[where[x]]) {
                swap(c[pos[where[x]]], c[R[f[where[x]]]]);
                pos[c[pos[where[x]]]] = pos[where[x]];
                pos[where[x]] = R[f[where[x]]];
                L[f[where[x]] + 1] = R[f[where[x]]]--;
                if (R[f[where[x]]] < L[f[where[x]]]) R[f[where[x]]] = -1;
                if (R[f[where[x]] + 1] == -1) R[f[where[x]] + 1] = L[f[where[x]] + 1];
                ++f[where[x]];
            }
            else ++cnt;
        ans = max(ans, cnt);
    }
    printf("%d\n", ans);
}
\end{lstlisting}
